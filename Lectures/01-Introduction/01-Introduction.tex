\documentclass{beamer}

\mode<presentation>
{
  \setbeamertemplate{background canvas}[square]
  \pgfdeclareimage[width=6em,interpolate=true]{dsailogo}{../dsai-logo}
  \pgfdeclareimage[width=6em,interpolate=true]{erasmuslogo}{../erasmus-logo}
  \titlegraphic{\pgfuseimage{dsailogo} \hspace{0.2in} \pgfuseimage{erasmuslogo}}
  %\usetheme{default}
  \usetheme{Madrid}
  \usecolortheme{rose}
  \usefonttheme[onlysmall]{structurebold}
}

\usepackage{pgf,pgfarrows,pgfnodes,pgfautomata,pgfheaps,pgfshade}
\usepackage{amsmath,amssymb}
\usepackage{graphics}
\usepackage{ragged2e}
\usepackage{array}
\usepackage[latin1]{inputenc}
\usepackage{colortbl}
\usepackage[absolute,overlay]{textpos}
\setlength{\TPHorizModule}{30mm}
\setlength{\TPVertModule}{\TPHorizModule}
\textblockorigin{10mm}{10mm}
\usepackage[english]{babel}
\usepackage{listings}
\setbeamercovered{dynamic}

\AtBeginSection[]{
  \begin{frame}<beamer>
  \frametitle{Outline}
  \tableofcontents[currentsection]
  \end{frame}
}

\title[Machine Learning]{Machine Learning\\Introduction}
\author{dsai.asia}
\institute[]{Asian Data Science and Artificial Intelligence Master's Program}
\date{}

% My math definitions

\renewcommand{\vec}[1]{\boldsymbol{#1}}
\newcommand{\mat}[1]{\mathtt{#1}}
\newcommand{\ten}[1]{\mathcal{#1}}
\newcommand{\crossmat}[1]{\begin{bmatrix} #1 \end{bmatrix}_{\times}}
\renewcommand{\null}[1]{{\cal N}(#1)}
\newcommand{\class}[1]{{\cal C}_{#1}}
\def\Rset{\mathbb{R}}
\def\Pset{\mathbb{P}}
\DeclareMathOperator*{\argmax}{argmax}
\DeclareMathOperator*{\argmin}{argmin}
\DeclareMathOperator*{\sign}{sign}
\DeclareMathOperator*{\cov}{Cov}
\def\norm{\mbox{$\cal{N}$}}

\newcommand{\stereotype}[1]{\guillemotleft{{#1}}\guillemotright}

\newcommand{\myfig}[3]{\centerline{\includegraphics[width={#1}]{{#2}}}
    \centerline{\scriptsize #3}}

\begin{document}

%%%%%%%%%%%%%%%%%%%%%%%%%%%%%%%%%%%%%%%%%%%%%%%%%%%%%%%%%%%%
%%             CONTENTS START HERE

%\setbeamertemplate{navigation symbols}{}

\frame{\titlepage}

%======================================================================
\section{Introduction}
%======================================================================

\begin{frame}{Introduction}{What is machine learning?}

  Machine learning is now near the top of the list of skills
  U.S.\ companies want to see in the people they hire.

  \medskip
  
  What's all the fuss, and what is machine learning?

  \medskip
  
  Many tasks we want computers to do are difficult to program
  directly.

  \medskip
  
  Examples: image recognition, speech recognition, controlling a
  self-driving car.

  \medskip

  \begin{block}{Machine learning}
  A set of tools that let us specify the computer's
  behavior by giving examples of \alert{how} it should respond in
  given situations, \alert{without specifying the computation
    necessary} to formulate that response.
  \end{block}

  \medskip
  
  We tell the computer \alert{what} it should decide to do in a
  situation but not \alert{how} to make the decision.

\end{frame}


\begin{frame}{Introduction}{What's a model?}

  Essential idea: we want to create a \alert{model} from data that can
  later be \alert{queried} when new situations arise.

  \medskip

  \begin{block}{Model}

    A (mathematical) function whose input is a \alert{description of
      the current situation} and whose output is a \alert{decision,
      recommendation, or action}.

  \end{block}

\end{frame}


\begin{frame}{Introduction}{Examples of ML in real life}
  
  We are using machine learning every time we
  \begin{itemize}
  \item Use a credit card
  \item Get a recommendation from Netflix or Amazon
  \item Ask Google for directions by voice
  \item Take a ride in our Tesla!
  \end{itemize}

  \medskip
  
  Let's brainstorm about things closer to home that might be using
  machine learning already or might benefit from it in the near
  future.

\end{frame}


\begin{frame}{Introduction}{The four problems for machine learning}

  Machine learning comprises perhaps four basic problems:
  \begin{itemize}
  \item \alert{Classification}: place instances into one or more of a
    set of given discrete \alert{categories}.
  \item \alert{Regression}: estimate a function from sample
    inputs/outputs that can later be used for \alert{interpolation} or
    \alert{extrapolation}.
  \item \alert{Density estimation}: estimate a probability density
    function from a sample from the distribution that can later be used,
    e.g., for \alert{anomaly detection}.
  \item \alert{Reinforcement}: derive a \alert{policy} that enables an
    agent to behave optimally in an uncertain environment using
    \alert{feedback} on the goodness of the outcome over time.
  \end{itemize}

  \medskip

  Let's think about the input and output of the model in each of these
  cases.
  
\end{frame}

\end{document}

