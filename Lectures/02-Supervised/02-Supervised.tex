\documentclass{beamer}

\mode<presentation>
{
  \setbeamertemplate{background canvas}[square]
  \pgfdeclareimage[width=6em,interpolate=true]{dsailogo}{../dsai-logo}
  \pgfdeclareimage[width=6em,interpolate=true]{erasmuslogo}{../erasmus-logo}
  \titlegraphic{\pgfuseimage{dsailogo} \hspace{0.2in} \pgfuseimage{erasmuslogo}}
  %\usetheme{default}
  \usetheme{Madrid}
  \usecolortheme{rose}
  \usefonttheme[onlysmall]{structurebold}
}

\usepackage{pgf,pgfarrows,pgfnodes,pgfautomata,pgfheaps,pgfshade}
\usepackage{amsmath,amssymb}
\usepackage{graphics}
\usepackage{ragged2e}
\usepackage{array}
\usepackage[latin1]{inputenc}
\usepackage{colortbl}
\usepackage[absolute,overlay]{textpos}
\setlength{\TPHorizModule}{30mm}
\setlength{\TPVertModule}{\TPHorizModule}
\textblockorigin{10mm}{10mm}
\usepackage[english]{babel}
\usepackage{listings}
\setbeamercovered{dynamic}

\AtBeginSection[]{
  \begin{frame}<beamer>
  \frametitle{Outline}
  \tableofcontents[currentsection]
  \end{frame}
}

\title[Machine Learning]{Machine Learning\\Supervised Learning}
\author{dsai.asia}
\institute[]{Asian Data Science and Artificial Intelligence Master's Program}
\date{}

% My math definitions

\renewcommand{\vec}[1]{\boldsymbol{#1}}
\newcommand{\mat}[1]{\mathtt{#1}}
\newcommand{\ten}[1]{\mathcal{#1}}
\newcommand{\crossmat}[1]{\begin{bmatrix} #1 \end{bmatrix}_{\times}}
\renewcommand{\null}[1]{{\cal N}(#1)}
\newcommand{\class}[1]{{\cal C}_{#1}}
\def\Rset{\mathbb{R}}
\def\Pset{\mathbb{P}}
\DeclareMathOperator*{\argmax}{argmax}
\DeclareMathOperator*{\argmin}{argmin}
\DeclareMathOperator*{\sign}{sign}
\DeclareMathOperator*{\cov}{Cov}
\def\norm{\mbox{$\cal{N}$}}

\newcommand{\stereotype}[1]{\guillemotleft{{#1}}\guillemotright}

\newcommand{\myfig}[3]{\centerline{\includegraphics[width={#1}]{{#2}}}
    \centerline{\scriptsize #3}}

\begin{document}

%%%%%%%%%%%%%%%%%%%%%%%%%%%%%%%%%%%%%%%%%%%%%%%%%%%%%%%%%%%%
%%             CONTENTS START HERE

%\setbeamertemplate{navigation symbols}{}

\frame{\titlepage}

%--------------------------------------------------------------------
%\part<presentation>{Part name}
%
%\frame{\partpage}

\begin{frame}
\frametitle{Readings}

Readings for these lecture notes:
\begin{itemize}
\item[-] Bishop, C. (2006), \textit{Pattern Recognition and Machine Learning},
  Springer, Chapters 3, 4, 6, 7.
\item[-] Hastie, T., Tibshirani, R., and Friedman, J. (2016),
  \textit{Elements of Statistical Learning: Data Mining, Inference, and
    Prediction}, Springer, Chapters 2, 3, 4, 12.
\item[-] Ng, A. (2017), \textit{Supervised Learning}, Lecture note set
  1 for CS229, Stanford University.
\item[-] Ng, A. (2017), \textit{Generative Learning Algorithms},
  Lecture note set 2 for CS229, Stanford University.
\item[-] Ng, A. (2017), \textit{Support Vector Machines},
  Lecture note set 3 for CS229, Stanford University.
\end{itemize}

These notes contain material $\copyright$ Bishop (2006), Hastie et
al.\ (2016), and Ng (2017).

\end{frame}

%======================================================================
\section{Linear regression}
%======================================================================

\begin{frame}{Linear regression}{Preliminaries}

  To see how a supervised learning model is formulated in detail, we
  begin with the simple case of \alert{linear regression}.

  \medskip
  
  Consider the simple example in which we would like to \alert{predict
    a person's weight} given his or her \alert{height} and
  \alert{age}.

  \medskip

  To set things up we need to specify
  \begin{itemize}
  \item The training set
  \item The stucture of the model
  \item A cost function
  \item A procedure for minimizing the cost function
  \end{itemize}

  \medskip

  We'll need some notation to get started.
  
\end{frame}


\begin{frame}{Linear regression}{Notation}

  Notation:
  \begin{itemize}
  \item A vector of input variables or \alert{features} for a given
    instance will be written $\vec{x}$ or $\vec{x}^{(i)}$, with $i \in
    \{1, \ldots, m\}$.
  \item The output or \alert{target} variable will be written $y$ or
    $y^{(i)}$.
  \item For a supervised learning problem, a \alert{training set} is a
    set of \alert{pairs} $\left\{(\vec{x}^{(i)},y^{(i)})\right\}_{i
      \in \{ 1, \ldots, m \}}$.
  \item The input space will be written ${\cal X}$, i.e.,
    $\vec{x}^{(i)} \in {\cal X}$.
  \item The target space will be written ${\cal Y}$, i.e.,
    $y^{(i)} \in {\cal Y}$.
  \end{itemize}

\end{frame}


\begin{frame}{Linear regression}{Training set}

  We need a collection of pairs $\left\{(\vec{x}^{(i)},y^{(i)})\right\}$.

  \medskip

  Somehow we have to collect data that look like

  \medskip

  \begin{center}
    \begin{tabular}{ccc}
      \textbf{Height} & \textbf{Age} & \textbf{Weight} \\
      \hline
      180 & 30 & 70 \\
      190 & 25 & 80 \\
      $\cdots$ & $\cdots$ & $\cdots$ \\
    \end{tabular}
  \end{center}  

  \medskip

  To the extent possible, the training set population we sample from
  should be representative of the population we will use the trained
  model on later.

  \medskip

  This is an important point that will come up later.
  
\end{frame}


\begin{frame}{Linear regression}{Structure of the model}

  If we have a simple problem where we're predicting a single real
  value (weight) given a single real value (height), then ${\cal X} =
  {\cal Y} = \Rset$.

  \medskip

  If we have two inputs (height and age) then ${\cal X} = \Rset^2$.

  \medskip

  In general, we may have $n$ features representing each input instance,
  so ${\cal X} = \Rset^n$.

  \medskip

  The model we learn will be a function $h : {\cal X} \mapsto {\cal Y}$.
  
\end{frame}


\begin{frame}{Linear regression}{Supervised learning problem}

  \begin{block}{The supervised learning problem}

    Given a training set, learn a function $h : {\cal X} \mapsto {\cal Y}$
    so that $h(\vec{x})$ is a ``good'' predictor of the corresponding value
    of $y$.
    
  \end{block}

  \medskip

  The function $h$ is called a \alert{hypothesis}.

  \medskip

  If $\cal Y$ is continuous, we have a \alert{regression} problem.

  \medskip

  If $\cal Y$ is a discrete set, we have a \alert{classification}
  problem.

\end{frame}


\begin{frame}{Linear regression}{Hypothesis space}

  Supervised machine learning algorithms can be characterized by the
  space of hypotheses $h : {\cal X} \mapsto {\cal Y}$ they search.

  \medskip

  When ${\cal X} = \Rset^n$ and ${\cal Y} = \Rset$ the simplest choice
  is to approximate $y$ as a linear function of $\vec{x}$:
  \[ h_{\vec{\theta}}(\vec{x}) = \theta_0 + \theta_1 x_1 + \cdots + \theta_n x_n .\]

  \medskip

  The $\theta_i$'s are \alert{parameters} or \alert{weights}
  parameterizing the space of linear mappings from $\cal X$ to $\cal
  Y$.

  \medskip

  It is convenient to introduce a dummy input variable $x_0=1$ so that
  we can write

  \[ h_{\vec{\theta}}(\vec{x}) = \sum_{i=0}^n \theta_i x_i = \vec{\theta}^\top \vec{x} \]
  
\end{frame}

\begin{frame}{Linear regression}{Cost function}

  How to find a good $h$, or in the case of linear regression, a good
  $\vec{\theta}$?

  \medskip

  In supervised learning, we require a \alert{cost function} that assigns
  a large cost to ``bad'' predictions $h(\vec{x})$ and a small
  cost to ``good'' predictions $h(\vec{x})$.

  \medskip

  For regression problems, we often used the \alert{least squares}
  cost function
  \[ J(\vec{\theta}) =
  \frac{1}{2}\sum_{i=1}^m (h_{\vec{\theta}}(\vec{x}^{(i)}) - y^{(i)})^2 .\] 

  \medskip

  Convince yourself that this choice of $J$ gives small values for
  good $h$ and large values for bad $h$.

  \medskip

  Beyond this intuition, we will see later that least squares makes a
  lot of sense when we formulate a \alert{generative probabilistic
    model} for the learning problem.
  
\end{frame}

\begin{frame}{Linear regression}{Minimizing the cost function}

  How to get a good $\vec{\theta}$?

  \medskip

  We need to find \[ \vec{\theta}^* = \argmin_{\vec{\theta}} J(\vec{\theta}) .\]

  \medskip

  This will come up again and again in the course!

  \medskip

  How to find the $\vec{\theta}$ minimizing $J(\vec{\theta})$?
  \begin{itemize}
  \item From your undergraduate probability and statistics course,
    perhaps, you know there is an analytical solution based on taking
    the gradient, setting the elements of the gradient to 0, and
    solving the resulting system of linear equations.
  \item More interesting for later in the course would be to begin
    with an initial guess about $\vec{\theta}$ and find the minimum
    using \alert{gradient descent}.
  \end{itemize}

  \medskip

  Exercise: find the gradient of $J(\vec{\theta})$, i.e.,
  $\nabla_J(\vec{\theta})$.
  
\end{frame}


\begin{frame}{Linear regression}{Minimizing the cost function}

  We have \[ \nabla_J(\vec{\theta}) = \begin{bmatrix}
    \frac{\partial J}{\partial \theta_0} &
    \frac{\partial J}{\partial \theta_1} &
    ... &
    \frac{\partial J}{\partial \theta_n} \end{bmatrix} \]

  You should be able to derive
  \begin{eqnarray}
    \frac{\partial J}{\partial \theta_0} & = & \sum_{i=1}^m (h_{\vec{\theta}}(\vec{x}^{(i)})-y^{(i)}) \nonumber \\
    \frac{\partial J}{\partial \theta_1} & = & \sum_{i=1}^m (h_{\vec{\theta}}(\vec{x}^{(i)})-y^{(i)}) x_1^{(i)} \nonumber \\
    \cdots & & \cdots \nonumber \\
    \frac{\partial J}{\partial \theta_n} & = & \sum_{i=1}^m (h_{\vec{\theta}}(\vec{x}^{(i)})-y^{(i)}) x_n^{(i)} \nonumber
  \end{eqnarray}

\end{frame}


\begin{frame}{Linear regression}{Minimizing the cost function}

  As we said, for an analytical solution we want to find the point
  where the gradient is 0.

  \medskip

  Setting the partial derivatives to 0 we obtain $n+1$ linear equations
  \begin{eqnarray}
    \sum_{i=1}^m (\theta_0 + \theta_1 x_1^{(i)} + \cdots + \theta_n x_n^{(i)} -y^{(i)}) & = & 0 \nonumber \\
    \sum_{i=1}^m (\theta_0 x_1^{(i)}+ \theta_1 x_1^{(i)}x_1^{(i)} + \cdots + \theta_n x_n^{(i)}x_1^{(i)} -y^{(i)}x_1^{(i)}) & = & 0 \nonumber \\
    \cdots & & \cdots \nonumber \\
    \sum_{i=1}^m (\theta_0 x_n^{(i)}+ \theta_1 x_1^{(i)} x_n^{(i)}+ \cdots + \theta_n x_n^{(i)}x_n^{(i)} -y^{(i)}x_n^{(i)}) & = & 0 \nonumber 
  \end{eqnarray}
  
\end{frame}


\begin{frame}{Linear regression}{Minimizing the cost function}

  With a bit more manipulation, we get the \alert{normal equations}
  \begin{eqnarray}
    \theta_0 m + \theta_1 \sum_{i=1}^mx_1^{(i)} + \cdots + \theta_n \sum_{i=1}^mx_n^{(i)} & = & \sum_{i=1}^m y^{(i)} \nonumber \\
    \theta_0 \sum_{i=1}^m x_1^{(i)}+ \theta_1 \sum_{i=1}^mx_1^{(i)}x_1^{(i)} + \cdots + \theta_n \sum_{i=1}^m x_n^{(i)}x_1^{(i)} & = & \sum_{i=1}^m y^{(i)}x_1^{(i)} \nonumber \\
    \cdots & & \cdots \nonumber \\
    \theta_0 \sum_{i=1}^m x_n^{(i)}+ \theta_1 \sum_{i=1}^mx_1^{(i)} x_n^{(i)}+ \cdots + \theta_n \sum_{i=1}^m x_n^{(i)}x_n^{(i)} & = & \sum_{i=1}^m y^{(i)}x_n^{(i)} \nonumber 
  \end{eqnarray}

  Take care to remember that the $x^{(i)}_j$ and $y{(i)}$ are all
  constant (they are given in the training set).

  \medskip

  This means we have $n+1$ linear equations in the $n+1$ unknown
  elements of $\vec{\theta}$.
  
\end{frame}


\begin{frame}{Linear regression}{Minimizing the cost function}

  Next, let's try to simplify our system of equations.
  If we write our training set in matrix and vector form as $\mat{X},
  \vec{y}$ with
  \[ \mat{X} = \begin{bmatrix}
    1 & x_1^{(1)} & \cdots & x_n^{(1)} \\
    1 & x_1^{(2)} & \cdots & x_n^{(2)} \\
    \vdots & \vdots & \ddots & \vdots \\ 
    1 & x_1^{(m)} & \cdots & x_n^{(m)} \end{bmatrix} \;\;\;
    \vec{y} = \begin{bmatrix}
    y^{(1)} \\ y^{(2)} \\ \vdots \\ y^{(m)}
    \end{bmatrix}, \]
    You should be able to verify that our system of linear equations
    becomes
    \[ \mat{X}^\top \mat{X} \vec{\theta} = \mat{X}^\top \vec{y}. \]
    This is a compact form of the normal equations. $\mat{X}$ is called
    the \alert{design matrix}.
    The parameters minimizing
    $J(\vec{\theta})$ must be
    \[ \vec{\theta} = (\mat{X}^\top \mat{X})^{-1}\mat{X}^\top \vec{y}. \]
       [Be careful about conditions under which $\mat{X}^\top \mat{X}$ has
         no inverse.]
\end{frame}


\begin{frame}{Linear regression}{Gradient descent}

  Now suppose we could find the gradient $\nabla_J(\vec{\theta})$ but
  were for some reason unable to solve for $\nabla_J(\vec{\theta}) = \vec{0}$.

  \medskip

  Whenever we are stuck without a closed form solution to an optimization
  problem, we can try to use iterative numerical methods.

  \medskip

  The simplest method is \alert{gradient descent}.

  \medskip

  The \alert{gradient} $\nabla_f(\vec{x})$ of a function $f(\vec{x})$
  evaluated at an arbitrary point $\vec{x}$ is a vector with two
  useful properties:
  \begin{itemize}
  \item The \alert{direction} of $\nabla_f(\vec{x})$ is the direction
    in which $f(\vec{x})$ is increasing the fastest.
  \item The \alert{magnitude} of $\nabla_f(\vec{x})$ is the slope
    of $f(\vec{x})$ in that direction of maximum increase.
  \end{itemize}

\end{frame}


\begin{frame}{Linear regression}{Gradient descent}

  If we start with a guess $\vec{\theta}^{(0)}$ and calculate the gradient
  $\nabla_J(\vec{\theta}^{(0)})$ at that guessed point, we know that the
  resulting vector points in the direction in which the cost function is
  increasing the most.

  \medskip

  Gradient descent simply repeats the process of moving $\vec{\theta}$
  in the direction \alert{opposite} $\vec{\theta}$ until we find the
  minimum of $J(\vec{\theta})$:
  
  \begin{tabbing}
    xxxx \= xxxx \= \kill
    $\vec{\theta} \leftarrow$ some initial guess \\
    While not converged \\
    \> $\vec{\theta}^{(n+1)} \leftarrow \vec{\theta}^{(n)} - \alpha \sum_{i=1}^m (h_{\vec{\theta^{(n)}}}(\vec{x}^{(i)}) - y^{(i)})\vec{x}^{(i)}$ 
  \end{tabbing}

  $\alpha$ (a small positive real number) is called the \alert{learning rate}.
  
\end{frame}


\begin{frame}{Linear regression}{Batch vs.\ stochastic gradient descent}
  \label{sgd}
  
  When we update our parameter vector $\vec{\theta}$ using the
  gradient calculated over the \alert{entire training set}, we are
  doing what is called \alert{batch} gradient descent.

  \medskip

  One alternative frequently used in deep learning is
  \alert{stochastic} gradient descent, in which we repeatedly update
  using the gradient calculated for \alert{each training element}:

  \begin{tabbing}
    xxxx \= xxxx \= \kill
    $\vec{\theta} \leftarrow$ some initial guess \\
    While not converged \\
    \> For $i \in 1..m$ \\
    \> \> $\vec{\theta}^{(n+1)} \leftarrow \vec{\theta}^{(n)} - \alpha (h_{\vec{\theta^{(n)}}}(\vec{x}^{(i)}) - y^{(i)})\vec{x}^{(i)}$ 
  \end{tabbing}

  \medskip

  Stochastic gradient descent tends to get close to the minimum faster
  than batch gradient descent but may \alert{oscillate} around the minimum.

  \medskip

  Slowly \alert{decreasing} the learning rate $\alpha$ over time can
  improve convergence.
  
\end{frame}


\begin{frame}{Linear regression}{Summary}

  OK! Now we have three methods for minimizing $J(\vec{\theta})$:
  \begin{itemize}
  \item Solve the \alert{normal equations}:
    \begin{tabbing}
      xxxx \= xxxx \= \kill
      $\vec{\theta} \leftarrow (\mat{X}^\top \mat{X})^{-1}\mat{X}^\top \vec{y}$
    \end{tabbing}
  \item \alert{Batch} gradient descent:
    \begin{tabbing}
      xxxx \= xxxx \= \kill
      $\vec{\theta} \leftarrow$ some initial guess \\
      While not converged \\
      \> $\vec{\theta}^{(n+1)} \leftarrow \vec{\theta}^{(n)} - \alpha \sum_{i=1}^m (h_{\vec{\theta^{(n)}}}(\vec{x}^{(i)}) - y^{(i)})\vec{x}^{(i)}$ 
    \end{tabbing}
  \item \alert{Stochastic} gradient descent:
    \begin{tabbing}
      xxxx \= xxxx \= \kill
      $\vec{\theta} \leftarrow$ some initial guess \\
      While not converged \\
      \> For $i \in 1..m$ \\
      \> \> $\vec{\theta}^{(n+1)} \leftarrow \vec{\theta}^{(n)} - \alpha (h_{\vec{\theta^{(n)}}}(\vec{x}^{(i)}) - y^{(i)})\vec{x}^{(i)}$
    \end{tabbing}    
  \end{itemize}

\end{frame}


\begin{frame}{Linear regression}{First Python tutorial}

  In Lab 01, you'll see how we can translate the mathematics
  presented here into working Python code.

\end{frame}


\begin{frame}{Linear regression}{Probabilistic interpretation}

  Last question about linear regression: \alert{why least squares}?
  Why not some other cost function?

  \medskip

  The least squares method comes naturally from a probabilistic
  interpretation of the problem.

  \medskip

  We suppose that our measurements $(\vec{x}^{(i)},y^{(i)})$ were
  \alert{generated} according to
  \[ y^{(i)} = \vec{\theta}^\top \vec{x}^{(i)} + \epsilon^{(i)}, \]
  where $\epsilon^{(i)}$ is an \alert{error} term representing noise
  and whatever effects our linear model doesn't capture.

\end{frame}


\begin{frame}{Linear regression}{Probabilistic interpretation}

  We might assume that the noise terms $\epsilon^{(i)}$
  were sampled i.i.d.\footnote{Independently and identically distributed}
  from some distribution $p(\epsilon^{(i)})$.

  \medskip

  [Is i.i.d.\ a reasonable assumption?]

  \medskip

  In many applications, a natural choice of $p(\epsilon^{(i)})$ is a
  \alert{Gaussian} (normal distribution), which we write
  \[ \epsilon^{(i)} \sim {\cal N}(0, \sigma^2). \]

  \medskip

  The univariate zero-mean Gaussian distribution is written
  \[ p(\epsilon^{(i)}) = \frac{1}{\sqrt{2\pi\sigma^2}} e^{\left( -\frac{(\epsilon^{(i)})^2}{2\sigma^2} \right) .} \]
  
\end{frame}


\begin{frame}{Linear regression}{Probabilistic interpretation}

  We would like to know $p(y^{(i)} \mid \vec{x}^{(i)} ; \vec{\theta})$.

  \medskip

  [This is the density over possible $y^{(i)}$ (weight measurements)
    we would obtain once we've identified $\vec{x}^{(i)}$ (the
    individual's height and age).]

  \medskip

  Recall that the density over $\epsilon^{(i)}$ is
  \[ p(\epsilon^{(i)}) = \frac{1}{\sqrt{2\pi\sigma^2}}
     e^{-\left(\frac{(\epsilon^{(i)})^2}{2\sigma^2}\right)}. \]

  \medskip
  
  Since $\epsilon^{(i)} = y^{(i)} - \vec{\theta}^\top\vec{x}^{(i)}$,
  the density
  over $y^{(i)}$ is just a shifted version of the density over $\epsilon^{(i)}$:
  \[ p(y^{(i)} \mid \vec{x}^{(i)} ; \vec{\theta} ) = \frac{1}{\sqrt{2\pi\sigma^2}} e^{\left( -\frac{(y^{(i)}-\vec{\theta}^\top\vec{x}^{(i)})^2}{2\sigma^2} \right) .} \]

  \medskip

  Note: the semicolon (;) in the expression $p(y^{(i)} \mid
  \vec{x}^{(i)} ; \vec{\theta} )$ means that the distribution is
  \alert{parameterized} by constant $\vec{\theta}$. $\vec{\theta}$ is
  not a random variable.
   
\end{frame}


\begin{frame}{Linear regression}{Probabilistic interpretation}

  Now consider the distribution of the vector of targets $\vec{y}$
  given the \alert{design matrix} $\mat{X}$. We write this
  \[ p(\vec{y} \mid \mat{X} ; \vec{\theta}) .\]

  \medskip

  Since the elements of $\vec{y}$ are independent, we can write
  \[ p(\vec{y} \mid \mat{X} ; \vec{\theta}) = \prod_{i=1}^m
  p(y^{(i)} \mid \vec{x}^{(i)} ; \vec{\theta} ) \]

  \medskip

  Since $\vec{y}$, $\mat{X}$, and $\vec{\theta}$ are all constant,
  we are simply multiplying the ``heights'' of $m$ Gaussians at the $y^{(i)}$'s.

  \medskip

  The Gaussians are centered at our \alert{predictions} of the
  $y^{(i)}$'s, so the conditional probability $p(\vec{y} \mid \mat{X}
  ; \vec{\theta})$ will be \alert{maximized} when our predictions of
  the $y^{(i)}$'s are \alert{perfect}.
  
  \medskip

  This means \alert{choosing $\vec{\theta}$ to maximize $p(\vec{y} \mid
  \mat{X} ; \vec{\theta})$} would be good!

\end{frame}


\begin{frame}{Linear regression}{Probabilistic interpretation}

  Now we think we would like to choose $\vec{\theta}$ to maximize
  $p(\vec{y} \mid \mat{X} ; \vec{\theta})$.

  \medskip

  First, let's write this distribution as a \alert{function} of the
  parameter vector $\vec{\theta}$:
  \[ L(\vec{\theta}) = L(\vec{\theta} ; \mat{X}, \vec{y}) = p(\vec{y} \mid \mat{X} ; \vec{\theta}) \]

  \medskip

  $L(\vec{\theta})$ is called the \alert{likelihood} function.

  \begin{block}{Maximum likelihood}
    The \alert{maximum likelihood} principle states that we should choose
    the $\vec{\theta}$ that makes the likelihood $L(\vec{\theta})$ as large
    as possible:
    \[ \vec{\theta}^* = \argmax_{\vec{\theta}} L(\vec{\theta}). \]
  \end{block}
  
  OK, let's try to maximize $L(\vec{\theta})$.
  
\end{frame}


\begin{frame}{Linear regression}{Probabilistic interpretation}

  As we already noted, since by assumption the $\epsilon^{(i)}$'s are
  independent,

  \begin{eqnarray}
    L(\theta) & = & \prod_{i=1}^m p(y^{(i)} \mid \vec{x}^{(i)} ; \vec{\theta}) \nonumber \\
    & = & \prod_{i=1}^m \frac{1}{\sqrt{2\pi\sigma^2}}\exp{\left( - \frac{(y^{(i)} - \vec{\theta}^\top \vec{x}^{(i)})^2}{2\sigma^2}\right)}. \nonumber
  \end{eqnarray}

  This looks difficult to maximize directly (try to evalute
  $\frac{\partial{L}}{\partial{\vec{\theta}}}$).

  \medskip

  However, we can equivalenently maximize \alert{any function that is
    strictly increasing} in $L(\vec{\theta})$.

\end{frame}


\begin{frame}{Linear regression}{Probabilistic interpretation}

  It will be more convenient to maximize the
  \alert{log likelihood} $\ell(\vec{\theta})$:

  \begin{eqnarray}
    \ell(\vec{\theta}) & = & \log L(\vec{\theta}) \nonumber \\
    & = & \log \prod_{i=1}^m p(y^{(i)} \mid \vec{x}^{(i)} ; \vec{\theta}) \\
    & = & \sum_{i=1}^m \log \frac{1}{\sqrt{2\pi\sigma^2}}\exp{\left( - \frac{(y^{(i)} - \vec{\theta}^\top \vec{x}^{(i)})^2}{2\sigma^2}\right)} \nonumber \\
    & = & m \log \frac{1}{\sqrt{2\pi\sigma^2}} - \frac{1}{2\sigma^2}\sum_{i=1}^m(y^{(i)} - \vec{\theta}^\top\vec{x}^{(i)})^2. \nonumber
  \end{eqnarray}

\end{frame}


\begin{frame}{Linear regression}{Probabilistic interpretation}

  Clearly, any $\vec{\theta}$ maximizing
  \[ m \log \frac{1}{\sqrt{2\pi\sigma^2}} - \frac{1}{2\sigma^2}\sum_{i=1}^m(y^{(i)} - \vec{\theta}^\top\vec{x}^{(i)})^2 \]
  will also maximize
  \[ - \frac{1}{2}\sum_{i=1}^m(y^{(i)} - \vec{\theta}^\top\vec{x}^{(i)})^2 .\]

  \bigskip

  Amazingly, maximizing $L(\vec{\theta})$ or $\ell(\vec{\theta})$
  gives us the same $\vec{\theta}$ we get by minimizing
  \[ J(\theta) = \frac{1}{2}\sum_{i=1}^m(y^{(i)}-\theta^\top\vec{x}^{(i)})^2 .\]

\end{frame}


\begin{frame}{Linear regression}{Probabilistic interpretation}

  We thus see that for linear regression, \alert{least squares and
    maximum likelihood are equivalent}.

  \medskip

  The equivalence comes from the assumption of i.i.d.\ Gaussian errors
  in our samples $y^{(i)}$.
  
  \medskip

  For other problems, we may not be able to formulate a least squares
  cost function, but we will be able to use the more general principle of
  maximum likelihood.

\end{frame}


\begin{frame}{Linear regression}{Summary of terms}

  
  \begin{tabular}{>{\raggedright}p{0.2\textwidth}>{\raggedright}p{0.2\textwidth}>{\raggedright\arraybackslash}p{0.5\textwidth}}
    \hline
    \textbf{Term} & \textbf{Definition / Symbol} & \textbf{Example} \\
    \hline
    Training set & $(\vec{x}^{(i)},y^{(i)})_{i \in 1..m}$ & Table of heights, ages, and weights \\
    \hline
    Features & $\vec{x}, \vec{x}^{(i)} \in {\cal X}$ & Height, age ($\Rset^2$) \\
    \hline
    Target & $y, y^{(i)} \in {\cal Y}$ & Weight ($\Rset$) \\
    \hline
    Hypothesis & $h_{\vec{\theta}} : {\cal X} \mapsto {\cal Y}$ & $y = \theta_0 + \theta_1 x_1 + \theta_2 x_2$ \\
    \hline
    Parameters or weights & $\vec{\theta}$ & $\theta_0$ (intercept), $\theta_1$ (kgs per cm), $\theta_2$ (kgs per year) \\
    \hline
    Cost function & $J(\vec{\theta})$ & $J(\vec{\theta}) = \frac{1}{2}\sum_{i=1}^m(h_{\vec{\theta}}(\vec{x}^{(i)} - y^{(i)})^2$ or negative log likelihood $\ell(\vec{\theta})$ \\
    \hline
  \end{tabular}

\end{frame}


\begin{frame}{Linear regression}{Summary of terms}

  \begin{tabular}{>{\raggedright}p{0.2\textwidth}>{\raggedright}p{0.2\textwidth}>{\raggedright\arraybackslash}p{0.5\textwidth}}
    \hline
    \textbf{Term} & \textbf{Definition / Symbol} & \textbf{Example} \\
    \hline
    Design matrix & $\mat{X}$ & $\mat{X} = \begin{bmatrix} 1 & x^{(1)}_1 & x^{(1)}_2 \\ \vdots & \vdots & \vdots \end{bmatrix}$ \\
    \hline
    Normal equations & $\mat{X}^\top\mat{X}\theta = \mat{X}^\top\vec{y}$ & \\
    \hline
    Batch gradient descent & $\vec{\theta}^{n+1} \leftarrow \vec{\theta}^{n} - \alpha \nabla_J(\vec{\theta}^{(n)})$ &  $\vec{\theta}^{(n+1)} \leftarrow \vec{\theta}^{(n)} - \alpha \sum_{i=1}^m (h_{\vec{\theta^{(n)}}}(\vec{x}^{(i)}) - y^{(i)})\vec{x}^{(i)}$ \\
    \hline
    Stochastic gradient descent &
    $\vec{\theta}^{n+1} \leftarrow \vec{\theta}^{n} - \alpha \frac{\partial (h_{\vec{\theta}^n}(\vec{x}^{(i)}) - y^{(i)})^2}{\partial \vec{\theta}^n}$
    & $\vec{\theta}^{(n+1)} \leftarrow \vec{\theta}^{(n)} - \alpha (h_{\vec{\theta^{(n)}}}(\vec{x}^{(i)}) - y^{(i)})\vec{x}^{(i)}$
    \\
    \hline
    Learning rate & $\alpha$ & 0.000001 \\
    \hline
    Gaussian/normal distribution & ${\cal N}(\mu,\sigma^2)$ & ${\cal N}(y^{(i)} - h_{\vec{\theta}}(\vec{x}^{(i)}), \sigma^2)$ \\
    \hline
    
  \end{tabular}


  
\end{frame}


\begin{frame}{Linear regression}{Overfitting and underfitting}

  So far we have only considered the linear model
  \[ h_{\vec{\theta}}(\vec{x}) = \vec{\theta}^\top\vec{x} .\]

  \medskip

  What if $y$ is not actually linearly related to $\vec{x}$?

  \medskip

  A simple trick is to add features that are nonlinear in $\vec{x}$.
  For example, if we have just one feature $x$, we add a new feature $x^2$:
  \[ y = \theta_0 + \theta_1 x + \theta_2 x^2 . \]
  The model is still linear in $\vec{\theta}$.

\end{frame}


\begin{frame}{Linear regression}{Overfitting and underfitting}

  We can take this further, for example to degree 5:
  \[ y = \sum_{j=0}^5 \theta_j x^j .\]

\end{frame}


\begin{frame}{Linear regression}{Overfitting and underfitting}

  Increasing the degree of $h_{\vec{\theta}}(\vec{x})$ works even if
  we have more than one feature ($n > 1$).

  \medskip
  
  With two features and polynomial degree 2:
  \[ y = \theta_0 + \theta_1 x_1 + \theta_2 x_2 + \theta_3 x_1^2 + \theta_4 x_2^2 + \theta_5 x_1 x_2 . \]

  \medskip

  The question is how far we should take this?

\end{frame}


\begin{frame}{Linear regression}{Overfitting and underfitting}

  It turns out that we can usually improve prediction performance on
  the training set by increasing the complexity of our model.

  \medskip
  
  What is the \alert{right} complexity for a model?

  \medskip

  We will return to this question later in the course, but for now,
  we introduce a simple technique:

  \medskip
  
  Instead of acquiring a \alert{single training set} $X =
  (\vec{x}^{(i)},y^{(i)})_{i \in 1..m}$, acquire two data sets:
  \begin{itemize}
  \item The training set $X^{(train)}$ consisting of $m_1$ pairs, used to
    estimate the optimal parameters $\vec{\theta}$ for $h_{\vec{\theta}}(\vec{x})$.
  \item A separate \alert{test set} $X^{(test)}$ consisting of $m_2$ pairs,
    used to check the \alert{generalization} ability of the
    resulting $h_{\vec{\theta}}(\vec{x})$.
  \end{itemize}

\end{frame}


\begin{frame}{Linear regression}{Overfitting and underfitting}

  As we increase the complexity of $h_{\vec{\theta}}(\vec{x})$,
  the cost $J(\vec{\theta})$ evaluated on the training set will usually
  decrease monotonically.

  \medskip

  However, we will also take the optimal $\vec{\theta}$ for the training set
  and calculate the same cost function $J(\vec{\theta})$ over the test set.

  \medskip

  Good heuristics for finding the right model complexity given
  separate training and test sets:
  \begin{itemize}
  \item When both training and test cost is high, we are \alert{underfitting}
    the data. Increase complexity.
  \item When training error is decreasing but test error is
    increasing, we are \alert{overfitting} the training data. Decrease
    complexity.
  \end{itemize}
  
\end{frame}


\begin{frame}{Linear regression}{Overfitting and underfitting}

  \alert{Exercise}: generate a height/weight dataset in which there is
  a \alert{quadratic} relationship between height and weight. Show
  that the linear model \alert{underfits} the data and that higher
  order polynomial fits \alert{overfit} the data.

  \medskip

  For example, you might use $y = \theta_0 + \theta_1 x + \theta_2 x^2
  + \epsilon$ and $\vec{\theta} = \begin{bmatrix} -426 & 5.31 &
    -0.0139 \end{bmatrix}$.

  \medskip

  Check that the overfitting problem gets worse as the size of
  the training set gets smaller or the complexity of the model
  increases.
  
\end{frame}


\begin{frame}{Linear regression}{Overfitting and underfitting}

  See the handout on the course Web page for detailed code and results.

  \medskip

  Here is the gist:

  \medskip

  \myfig{4in}{over-under}{}
  
\end{frame}


\begin{frame}{Linear regression}{Locally weighted linear regression}

  If you play with the overfitting and underfitting tutorial notebook,
  you'll find that the test set heuristic is not always perfectly reliable.

  \medskip

  Even when we know that the ``ground truth'' model is quadratic, the
  test set might happen to be best fit by a higher-complexity model,
  depending on the training and test set distributions.

  \medskip

  Is there any other (hopefully more reliable) way to avoid
  overfitting?

  \medskip

  One general approach that we'll see more of later is
  a \alert{non-paramteric} approach.

  \medskip

  The particular non-paramteric approach we'll explore today is
  \alert{locally weighted linear regression}.

\end{frame}


\begin{frame}{Linear regression}{Locally weighted linear regression}

  The basic approach is to define
  \[ h(\vec{x}) = \vec{\theta}^\top \vec{x}, \]
  where
  \[ \vec{\theta} = \argmin_{\vec{\theta}_l} \sum_{i=1}^m w^{(i)} (y^{(i)}-\vec{\theta}_l^\top \vec{x}^{(i)})^2 \]
  and $w^{(i)}$ is a non-negative \alert{weight} on training example
  $i$.

  \medskip

  $w^{(i)}$ controls how much training example $i$ influences $\vec{\theta}$.

  \medskip

  How to set $w^{(i)}$?
  
\end{frame}


\begin{frame}{Linear regression}{Locally weighted linear regression}

  One choice of $w^{(i)}$ uses a Gaussian \alert{kernel} around $\vec{x}^{(i)}$:
  \[ w^{(i)} = e^{-\frac{(\vec{x}^{(i)}-\vec{x})^\top(\vec{x}^{(i)}-\vec{x})}{2\tau^2}} \]

  \medskip

  LWLR is a \alert{non-paramteric} method because it doesn't derive
  a complete fixed vector of parameters from the training set.

  \medskip

  Instead, we keep the \alert{entire training set} around, and when we
  want a prediction for a previously unseen $\vec{x}$, we do a local
  fit given the nearby training set items.

  \medskip

  The Gaussian kernel width $\tau$ is called the \alert{bandwidth}
  parameter, which affects how local LWLR is.

  \medskip

  The bandwidth parameter $\tau$ is our first example of a \alert{free
    parameter} or a \alert{hyperparameter}.
  
\end{frame}


\begin{frame}{Linear regression}{Locally weighted linear regression}
  
  \alert{Exercise}: derive the calculation of $\vec{\theta}$ given a
  set of weights $w^{(i)}$ over the training set.

  \medskip

  \alert{Hint}: factor the weights into a diagonal matrix $\mat{W}$ so
  that you end up with $\mat{W}$ in the right place on the left and
  right side of the normal equations.

  \medskip

  \alert{Exercise}: write a Python function that implements
  $h_{\mat{X},\vec{y},\tau}(\vec{x})$.
  
\end{frame}

%======================================================================
\section{Classification}
%======================================================================

\begin{frame}{Classification}
  
  Next we move to \alert{classification}.

  \medskip

  We still have a training set $(\vec{x}^{(i)},y^{(i)})_{i \in 1..m}$
  and a hypothesis $h_{\vec{\theta}}(\vec{x})$, but now the values
  $y^{(i)}$ can take on are a small number of discrete values.

  \medskip

  These discrete values could be \alert{unordered} (cat, dog, mouse)
  or \alert{ordered} (small, medium, large).

  \medskip

  We'll first consider \alert{binary classification} in which $y^{(i)}
  \in \{ 0, 1 \}$.

  \medskip
  
  Think of some examples of binary classification:
  \begin{itemize}
  \item Classifying an image as a cat or a dog.
  \item Classifying an email as spam or non-spam.
  \item Classifying a university degree as fake or legitimate.
  \item And on and on ...
  \end{itemize}

  \medskip

  Sometimes class 1 will be called \alert{positive} and class 0 will
  be called \alert{negative}.
  
\end{frame}


\begin{frame}{Classification}{Example}

  A great example of classification is deciding whether an email is
  \alert{spam or not spam}.

  \medskip

  Suppose we read thousands of email messages, manually divvyed them
  up into a \texttt{spam} and a \texttt{email} folder, and counted the
  \alert{frequency of each word and punctuation mark} in each message.

  \medskip

  We might get data like this for the words and characters with the
  largest average difference between \texttt{spam} and \texttt{email}
  (Hastie et al., 2016):

  \begin{center}
    \scriptsize
    \begin{tabular}{lccccccccccc}
      \hline & \texttt{george} & \texttt{you} & \texttt{your} &
      \texttt{hp} & \texttt{free} & \texttt{hp1} & \texttt{!} &
      \texttt{our} & \texttt{re} & \texttt{edu} & \texttt{remove}
      \\
      \hline
      spam & 0.00 & 2.26 & 1.38 & 0.02 & 0.52 & 0.01 & 0.51 &
      0.51 & 0.13 & 0.01 & 0.28 \\
      email & 1.27 & 1.27 & 0.44 & 0.90 & 0.07 & 0.43 & 0.11 & 0.18 & 0.42 & 0.29 & 0.01 \\ \hline
    \end{tabular}
  \end{center}

Each training vector $\vec{x}^{(i)}$ would contain the frequency of the tokens
\texttt{george}, \texttt{you},
and so on in one document. The corresponding $y^{(i)}$ might
be 0 for email and 1 for spam.

\end{frame}

%======================================================================
\section{Logistic regression}
%======================================================================

\begin{frame}{Logistic regression}{Hypothesis space}

  Given what we know so far, the simplest thing to do might be to
  ignore the fact that the $y^{(i)}$'s only come in two values and use
  linear regression with the values 0 and 1.

  \medskip

  Given a query value for $\vec{x}$, then, we might answer ``1'' if
  $h_{\vec{\theta}}(\vec{x}) \ge 0.5$ and ``0'' otherwise.
  
  \medskip

  This won't work very well!

  \medskip

  Improvement: let's change the form of $h$ to constrain
  it to the range $[0..1]$:
  \[ h_{\vec{\theta}}(\vec{x}) = g(\vec{\theta}^\top \vec{x}) = \frac{1}{1+e^{-\vec{\theta}^\top \vec{x}}}. \]
  Here the function
  \[ g(z) = \frac{1}{1 + e^{-z}} \]
  is called the \alert{logistic sigmoid function} or just the
  \alert{sigmoid function}.

\end{frame}


\begin{frame}[fragile]
  \frametitle{Logistic regression}
  \framesubtitle{Logistic sigmoid}

 \begin{columns}

 \column{2in}

\begin{tiny}
\begin{lstlisting}
import numpy
import matplotlib.pyplot as plt

def f(z):
    return 1/(1+numpy.exp(-z))

z = numpy.arange(-5, 5, 0.1)
plt.plot(z, f(z), 'b')
plt.xlabel('z')
plt.ylabel('g(z)')
plt.title('Logistic sigmoid function')
\end{lstlisting}
\end{tiny}

\column{2in}

\myfig{2.5in}{logistic}{}
  
  \end{columns}

\end{frame}


\begin{frame}{Logistic regression}{Derivative of the sigmoid function}

  To optimize a cost function involving the logistic sigmoid, we'll
  need to know its derivative. Here's a nice trick:

  \begin{eqnarray}
    g'(z) & = & \frac{d}{dz} \frac{1}{1+e^{-z}} \nonumber \\
    & = & \frac{1}{(1+e^{-z})^2}(e^{-z}) \nonumber \\
    & = & \frac{1}{1+e^{-z}} \cdot \left(\frac{1+e^{-z}-1}{1+e^{-z}}\right) \nonumber \\
    & = & g(z)(1-g(z)) \nonumber
  \end{eqnarray}
  
\end{frame}


\begin{frame}{Logistic regression}{Cost function}

  So far we have seen
  \begin{itemize}
  \item Training set. We are currently limiting ourselves to \alert{binary} classification.
  \item The form of the hypothesis $h_{\vec{\theta}}(\vec{x}) = g(\vec{\theta}^\top \vec{x})$.
  \end{itemize}

  What else do we need? A cost function and a method to minimize that cost.

  \medskip

  Since least squares doesn't make sense for classification, we'll
  instead \alert{maximize the likelihood} (actually the log likelihood)
  of $\vec{\theta}$.

  \medskip

  (Our cost function, then, will be negative log likelihood.)
  
\end{frame}


\begin{frame}{Logistic regression}{Likelihood function}

  The likelihood of $\vec{\theta}$ is
  \begin{eqnarray}
    L(\vec{\theta}) & = & p(\vec{y} \mid \mat{X} ; \vec{\theta}) \nonumber \\
    & = & \prod_{i=1}^m p(y^{(i)} \mid \vec{x}^{(i)} ; \vec{\theta}) .\nonumber
  \end{eqnarray}

  How should we model $p(y^{(i)} \mid \vec{x}^{(i)} ; \vec{\theta})$?

  \medskip

  It should be \alert{maximal} when our prediction
  $h_{\vec{\theta}}(\vec{x}^{(i)})$ is \alert{correct} and
  \alert{minimal} when the prediction is \alert{incorrect}.

\end{frame}


\begin{frame}{Logistic regression}{Likelihood function}

  In logistic regression we assume
  \[ p(y \mid \vec{x} ; \vec{\theta}) = \begin{cases}
    h_{\vec{\theta}}(\vec{x}) & y=1 \\
    1-h_{\vec{\theta}}(\vec{x}) & y=0 \\    
  \end{cases} \]
  That is, $h_{\vec{\theta}}(\vec{x})$ models $p(y=1 \mid \vec{x})$.

  \medskip

  [Discuss why this is reasonable.]

  \medskip

  A compact representation of the equation:
  \[ p(y \mid \vec{x} ; \vec{\theta}) = (h_{\vec{\theta}}(\vec{x}))^y(1-h_{\vec{\theta}}(\vec{x}))^{1-y} \]

\end{frame}


\begin{frame}{Logistic regression}{Likelihood function}

  We thus obtain
  \begin{eqnarray}
    L(\vec{\theta}) & = & \prod_{i=1}^m p(y^{(i)} \mid \vec{x}^{(i)} ; \vec{\theta}) \nonumber \\
    & = & \prod_{i=1}^m (h_{\vec{\theta}}(\vec{x}^{(i)}))^{y^{(i)}}(1-h_{\vec{\theta}}(\vec{x}^{(i)}))^{1-y^{(i)}} \nonumber
  \end{eqnarray}

  and the log likelihood is
  \begin{eqnarray}
    \ell(\vec{\theta}) & = & \log L(\vec{\theta}) \nonumber \\
    & = & \sum_{i=1}^m y^{(i)} \log h_{\vec{\theta}}(\vec{x}^{(i)}) + (1-y^{(i)})\log (1-h_{\vec{\theta}}(\vec{x}^{(i)})),
    \nonumber 
  \end{eqnarray}
  
\end{frame}


\begin{frame}{Logistic regression}{Maximizing the likelihood function}

  To find the maximum of $\ell(\vec{\theta})$, we first need to find
  $\nabla_{\ell}(\vec{\theta})$.

  \medskip

  We won't have a closed form solution for
  $\nabla_{\ell}(\vec{\theta}) = \vec{0}$ so instead we'll need to use
  gradient ascent.

  \medskip

  Let's first try the \alert{stochastic} version of gradient ascent in
  which we assume the training set is just a single pair
  $(\vec{x},y)$.

\end{frame}


\begin{frame}{Logistic regression}{Maximizing the likelihood function}

  \[ \frac{\partial}{\partial \theta_j} \ell(\vec{\theta}) =
  \left( y \frac{1}{g(\vec{\theta}^\top \vec{x})} - (1-y)\frac{1}{1-g(\vec{\theta}^\top \vec{x})} \right) \frac{\partial}{\partial \theta_j}g(\vec{\theta}^\top \vec{x}). \]

  Since
  \[ \frac{\partial}{\partial \theta_j}g(\vec{\theta}^\top \vec{x}) =
  g(\vec{\theta}^\top\vec{x})(1-g(\vec{\theta}^\top \vec{x})) \frac{\partial}{\partial \theta_j}(\vec{\theta}^\top \vec{x}), \]

  we obtain
  \begin{eqnarray}
    \frac{\partial}{\partial \theta_j} \ell(\vec{\theta}) & = &
    \left( y (1-g(\vec{\theta}^\top \vec{x})) - (1-y)g(\vec{\theta}^\top \vec{x}) \right) x_j \nonumber \\
    & = & (y-h_{\vec{\theta}}(\vec{x})) x_j .\nonumber
  \end{eqnarray}

  Therefore
  \[ \nabla_\ell(\vec{\theta}) = (y-h_{\vec{\theta}}(\vec{x}))\vec{x}. \]
    
\end{frame}


\begin{frame}{Logistic regression}{Stochastic gradient ascent rule}

  We thus get the stochastic gradient ascent rule for logistic regression:
  \[ \vec{\theta}^{(n+1)} \leftarrow \vec{\theta}^{(n)} + \alpha(y^{(i)} -h_{\vec{\theta}}(\vec{x}^{(i)}))\vec{x}^{(i)}. \]

  Does this look familiar? Take a look at page 23!

  \medskip

  Note that the update is not exactly the same, as
  $h_{\vec{\theta}}(\vec{x})$ is not the same.
  
  \medskip

  Still, it is amazing that we get similar update rules for
  \begin{itemize}
  \item Linear regression with least squares
  \item Linear regression with maximum likelihood
  \item Logistic regression with maximum likelihood
  \end{itemize}

  We will see how generalized linear models unify these seemingly
  different methods.
  
\end{frame}


\begin{frame}{Logistic regression}{Relationship to the perceptron}

  \begin{columns}

    \column{2.2in}

    \myfig{2in}{mark-1}{Mark I Perceptron Machine (Wikipedia)}

    \column{2.3in}
    
    In 1957, Rosenblatt conceived of the \alert{perceptron}, a
    physical machine implementing the classification function
    
    \[ h_{\vec{\theta}}(\vec{x}) = g(\vec{\theta}^\top \vec{x}) \]

    with

    \[ g(z) = \begin{cases} 1 & z \ge 0 \\ 0 & z < 0 \end{cases} .\]

  \end{columns}
  
\end{frame}


\begin{frame}{Logistic regression}{Relationship to the perceptron}

  The \alert{perceptron learning algorithm} also used the update rule
  \[ \vec{\theta}^{(n+1)} \leftarrow \vec{\theta}^{(n)} + \alpha(y^{(i)} -h_{\vec{\theta}}(\vec{x}^{(i)}))\vec{x}^{(i)}. \]

  \medskip

  Note that this is also different from the logistic and linear regression
  rules since $h_{\vec{\theta}}(\vec{x})$ is in this case a hard threshold
  classifier without any probabilistic interpretation.

\end{frame}


\begin{frame}{Logistic regression}{Relationship to the perceptron}

  Rosenblatt was extremely optimistic about the perceptron (the first
  ``neural network''), predicting it may in the future learn and
  translate languages.

  \medskip

  Minsky and Papert's (1969) \textit{Perceptrons} critiqued the power
  of the perceptron and helped kill neural network research for many
  years.

  \medskip

  Neural network research later went through two ``rebirths:''
  \begin{itemize}
  \item Rumelhart, Hinton, and Williams (1986): backpropagation
  \item Krizhevsky, Sutskever, and Hinton (2012): AlexNet
  \end{itemize}

  \medskip

  We're thus in the third cycle of possibly over-hyped expectations
  about neural networks!
  
\end{frame}


\begin{frame}{Logistic regression}{Netwon's method}

  We are now familiar with solving supervised learning problems through
  \begin{itemize}
  \item Direct solution of $\nabla_J(\vec{\theta}) =  \vec{0}$ or
    $\nabla_\ell(\vec{\theta}) =  \vec{0}$
  \item Gradient descent on a cost function $\vec{\theta} \leftarrow
    \vec{\theta} - \alpha \nabla_J(\vec{\theta})$
  \item Gradient ascent on a log likelihood function $\vec{\theta}
    \leftarrow \vec{\theta} + \alpha \nabla_\ell(\vec{\theta})$
  \end{itemize}

  We've seen that the gradient methods are not very efficient. Our
  jump will usually either \alert{undershoot} or \alert{overshoot}
  depending on the local gradient and our distance from the minimum.

  \medskip

  \alert{Newton's method} is a faster iterative method for finding the
  zero of an aribtrary nonlinear function.

\end{frame}


\begin{frame}{Logistic regression}{Netwon's method}

  The update rule in Netwon's method (for a 1D function) is
  \[ \theta^{(n+1)} \leftarrow \theta^{(n)} - \frac{f(\theta)}{f'(\theta)}. \]
  This corresponds to finding the \alert{intersection of the tangent to $f(\theta)$ with the line $y=0$}:

  \myfig{4in}{newton}{Ng, CS229 lecture note set \#1}

  \medskip

  See animated Newton's method examples online, e.g. at
  \url{http://www.cs.uleth.ca/~holzmann/notes/NewtonsMethod/}
  
\end{frame}


\begin{frame}{Logistic regression}{Netwon's method}

  Note, however, that in our optimization problems, we don't want to
  find a zero of the objective function; we want to find the
  \alert{minimum or maximum}.

  \medskip

  Luckily the problem is easily solved: \alert{find the minimum or maximum
  of $J(\vec{\theta})$ by finding the zero of $\nabla_J(\vec{\theta})$}.

  \medskip

  \alert{Exercise}: use Newton's method to find the minimum of
  \[ f(x) = (x-2)^2+1 \]
  beginning from $x_0 = 0$. How many iterations are required?
  Check by inspecting the function or solving $f'(x)=0$.

\end{frame}


\begin{frame}{Logistic regression}{Netwon's method}

  For the logistic regression problem, then, instead of slowly
  climbing the gradient of $\ell(\vec{\theta})$, we can use Newton's
  method to \alert{find the zero of $\nabla_\ell(\vec{\theta})$}.

  \medskip

  In 1D, we would have
  \[ \theta \leftarrow \theta - \frac{\ell'(\theta)}{\ell''(\theta)} .\]
  In multiple dimensions, the Newton-Raphson generalization of Newton's rule is
  \[ \vec{\theta} \leftarrow \vec{\theta} - (\mat{H}_\ell(\vec{\theta}))^{-1} \nabla_\ell(\vec{\theta}) .\]

  The matrix
  \[ \mat{H}_\ell(\vec{\theta}) = \begin{bmatrix}
    \frac{\partial^2}{\partial \theta_0 \theta_0}\ell(\vec{\theta}) &
    \frac{\partial^2}{\partial \theta_0 \theta_1}\ell(\vec{\theta}) &
    \cdots \\
    \frac{\partial^2}{\partial \theta_1 \theta_0}\ell(\vec{\theta}) &
    \frac{\partial^2}{\partial \theta_1 \theta_1}\ell(\vec{\theta}) &
    \cdots \\
    \vdots & \vdots & \ddots   
  \end{bmatrix}
  \]
  of second derivatives
  is called the \alert{Hessian} of $\ell(\vec{\theta})$.

\end{frame}


\begin{frame}{Logistic regression}{Netwon's method}

  \alert{Exercise}: recalling that for a one-pair training set,
  \[ \frac{\partial}{\partial \theta_j} \ell(\vec{\theta}) =
    (y-h_{\vec{\theta}}(\vec{x})) x_j ,\]
  find $\mat{H}_\ell(\vec{\theta})$.
  
%  \begin{eqnarray}
%    \frac{\partial}{\partial \theta_j \theta_k}\ell(\vec{\theta}) & = &
%    x_j x_k \frac{e^{\theta^\top \vec{x}}}{(1+e^{\theta^\top \vec{x}})^2} \nonumber \\
%    & = & x_j x_k \frac{1+e^{\theta^\top \vec{x}}-1}{(1+e^{\theta^\top \vec{x}})^2} \nonumber \\
%    & = & x_j x_k (h_{\vec{\theta}}(\vec{x}) - h_{\vec{\theta}}(\vec{x})^2) \nonumber \\
%    & = & x_j x_k h_{\vec{\theta}}(\vec{x}) (1 - h_{\vec{\theta}}(\vec{x})) \nonumber 
%  \end{eqnarray}

  \medskip
  
  In PS2, you will implement the batch Newton's method for logistic
  regression in Python.
  
\end{frame}

%======================================================================
\section{Generalized linear models}
%======================================================================

\begin{frame}{Generalized linear models}{Introduction}

  In \alert{linear regression}, we observe a random variable $y$
  assumed to be drawn from a Gaussian distribution depending linearly
  on a random vector $\vec{x}$ drawn from some population with
  conditional density
  \[ p(y \mid \vec{x} ; \vec{\theta}) = {\cal N}(\vec{\theta}^\top \vec{x}, \sigma^2) .\]

  \medskip

  In \alert{logistic regression}, we observe a random variable $y$
  assumed to be drawn from a Bernoulli distribution\footnote{Bernoulli
    just means $y$ is sampled so that $y=1$ with probability $p$ and
    $y=0$ with probability $1-p$.} depending on a random vector $\vec{x}$
  drawn from some population with conditional density
  \[ p(y \mid \vec{x} ; \vec{\theta}) = \textrm{Bernoulli}(g(\vec{\theta}^\top \vec{x})) .\]

  \medskip

  We'll see that these are both \alert{generalized linear models} (GLMs).

\end{frame}


\begin{frame}{Generalized linear models}{Exponential family}

  To understand GLMs we need to understand the \alert{exponential
    family} of distributions. Following Ng, a class of distributions
  is in the exponential family if it can be written in the form
  \[ p(y ; \vec{\eta}) = b(y) e^{(\vec{\eta}^\top T(y)-a(\vec{\eta}))},\]
  where
  \begin{itemize}
  \item $\vec{\eta}$ is the \alert{natural parameter} or
    \alert{canonical parameter} of the distribution,
  \item $T(y)$ is the \alert{sufficient statistic} (we normally use
    $T(y)=y$), and
  \item $a(\vec{\eta})$ is the \alert{log partition function}. We use
    $e^{-a(\vec{\eta})}$ just to normalize the distribution to have a
    sum or integral of 1.
  \item $b(y)$ is an arbitrary scalar function of $y$.
  \end{itemize}

  \medskip

  Each choice of $T$, $a$, and $b$ defines a \alert{family} (set) of
  distributions parameterized by $\vec{\eta}$.
 
\end{frame}


\begin{frame}{Generalized linear models}{Exponential family: Bernoulli}

  The Gaussian and Bernoulli distributions are both exponential family distributions.

  \medskip

  If $y$ is $\textrm{Bernoulli}(\phi)$, then $p(y = 1 ; \phi) = \phi$ and
  $p(y=0 ; \phi) = 1-\phi$.

  \medskip

  We can rewrite (using the substitution $z = e^{\log z}$)
  \begin{eqnarray}
    p(y ; \phi) & = & \phi^y (1-\phi)^{1-y} \nonumber \\
    & = & e^{(y \log \phi + (1-y)\log(1-\phi))} \nonumber \\
    & = & e^{(\log{\frac{\phi}{1-\phi}})y + \log{(1-\phi)}} \nonumber
  \end{eqnarray}

  assigning $\vec{\eta}=[\log{\frac{\phi}{1-\phi}}]$, $T(y)=y$,
  $a(\vec{\eta}) = \log(1+e^{\vec{\eta}})$, and $b(y)=1$, we see that
  $p(y ; \phi)$ is in the exponential family.
  
\end{frame}


\begin{frame}{Generalized linear models}{Exponential family: Gaussian}

  For the 1D Gaussian, assuming for now that $\sigma^2 = 1$, we obtain
  \begin{eqnarray}
    p(y ; \mu) & = & \frac{1}{\sqrt{2\pi}}e^{-\frac{1}{2}(y-\mu)^2} \nonumber \\
    & = & \frac{1}{\sqrt{2\pi}}e^{-\frac{1}{2}y^2}e^{\mu y - \frac{1}{2}\mu^2} \nonumber
  \end{eqnarray}

  If we assign $\eta = \mu$, $T(y) = y$, $a(\eta) = \mu^2/2 =
  \eta^2/2$, and $b(y) = (1/\sqrt{2\pi})e^{-y^2/2}$, we see that the
  Gaussian is in the exponential family as well.
  
\end{frame}


\begin{frame}{Generalized linear models}{Exponential family}

  Other members of the exponential family useful for machine learning:
  \begin{itemize}
  \item \alert{Multinomial}, for $n$-class classification problems
  \item \alert{Poisson}, for modeling count data
  \item \alert{Gamma} and \alert{exponential}, for continuous non-negative random variables like time intervals
  \item \alert{Beta} and \alert{Dirichlet} for distributions over probabilities
  \item and more...
  \end{itemize}

  If $y$ is in the exponential family given $\vec{x}$ and
  $\vec{\theta}$, we can apply the same procedure (the GLM recipe) to
  come up with a model.

\end{frame}


\begin{frame}{Generalized linear models}{Assumptions}

  The GLM makes three assumptions:
  \begin{enumerate}
  \item $p(y \mid \vec{x} ; \vec{\theta}) =
    \textrm{ExponentialFamily}(\vec{\eta})$.
  \item Given $\vec{x}$, we would like to predict an expected value
    of $T(y)$ given $\vec{x}$.
  \item $\vec{\eta}$ is linear in $\vec{x}$, i.e., $\eta_i = \vec{\theta}_i^\top \vec{x}$.
  \end{enumerate}

  Assumption 2 means that we want to learn a hypothesis function
  $h(\vec{x}) = E[y\mid \vec{x}].$

  \medskip

  For logistic regression, this would be
  \begin{eqnarray}
    h_{\vec{\theta}}(\vec{x}) & = & E[y \mid \vec{x} ; \vec{\theta} ] \nonumber \\
    & = & 0 \cdot p(y=0 \mid \vec{x}; \vec{\theta}) + 1 \cdot p(y = 1 \mid \vec{x};\vec{\theta}) \nonumber \\
    & = &     p(y=1 \mid \vec{x};\vec{\theta}) \nonumber
  \end{eqnarray}
  
\end{frame}


\begin{frame}{Generalized linear models}{Linear regression as a GLM}

  In the linear regression setting, if we apply the GLM assumptions with
  the Gaussian distribution, we obtain
  \[ y \sim {\cal N}(\mu, \sigma^2), \]
  and $h_{\vec{\theta}}(\vec{x})$ needs to be a prediction of $T(y)$
  given $\vec{x}$ and $\vec{\theta}$.

  \medskip
  
  We already found that the Gaussian is an exponential family distribution with
  natural parameter $\eta = \mu$.

  \medskip

  Since $\eta = \vec{\theta}^\top \vec{x}$, letting $T(y)=y$, we obtain
  \begin{eqnarray}
    h_{\vec{\theta}}(\vec{x}) & = & E[y \mid \vec{x} ; \vec{\theta} ] \nonumber \\
    & = & \mu \nonumber \\
    & = & \eta \nonumber \\
    & = & \vec{\theta}^\top \vec{x} .\nonumber
  \end{eqnarray}
  
\end{frame}


\begin{frame}{Generalized linear models}{Linear regression as a GLM}

  So now we have a new justification for the least squares approach to
  estimate a relationship between a continuous variable $y$ and a
  feature vector $\vec{x}$.

  \medskip

  If we assume $y$ given $\vec{x}$ is Gaussian with some mean $\mu$,
  the GLM approach states that we should use
  \[ h_{\vec{\theta}}(\vec{x}) = \vec{\theta}^\top \vec{x} .\]

  Then, given a training set $(\mat{X},\vec{y})$, we find that the
  $\vec{\theta}$ maximizing $p(\vec{y} \mid \mat{X} ; \theta)$ is
  $(\mat{X}^\top\mat{X})^{-1}\mat{X}^\top\vec{y}$.

\end{frame}


\begin{frame}{Generalized linear models}{Logistic regression as a GLM}

  Now consider the logistic regression setting. We have a two classes,
  0 and 1.

  \medskip

  If we assume $y \sim \textrm{Bernoulli}(\phi)$ and take the GLM
  approach, we cast the Bernoulli distribution as a member of the
  exponential family, obtaining
  \[ \phi = \frac{1}{1+e^{-\eta}}. \]

  We then try to predict the expectation of $T(y) = y$ given
  $\vec{x}$:
  \begin{eqnarray}
    h_{\vec{\theta}}(\vec{x}) & = & E[y \mid \vec{x} ; \vec{\theta}] \nonumber \\
    & = & \phi \nonumber \\
    & = & \frac{1}{1+e^{-\eta}} \nonumber \\
    & = & \frac{1}{1+e^{-\vec{\theta}^\top \vec{x}}} .\nonumber
  \end{eqnarray}
  Try to justify each step here!
   
\end{frame}


\begin{frame}{Generalized linear models}{Logistic regression as a GLM}

  Now we understand why we take the logistic sigmoid
  \[ \frac{1}{1+e^{-\vec{\theta}^\top\vec{x}}} \]
  as a model for $p(y=1 \mid \vec{x} ; \vec{\theta})$ in logistic
  regression: it is the natural consequence of \alert{choosing a GLM to model
  $y$ as a Bernoulli random variable depending on $\vec{x}$}.
  
\end{frame}


\begin{frame}{Generalized linear models}{Other models}

  Why should we care about all this? We already know how to do linear
  and logistic regression!!

  \medskip

  The reason is that the GLM approach applies to \alert{any
    distribution} and usually leads to \alert{elegant learning rules}.

  \medskip

  So when faced with a new learning problem, your baseline approach should be
  \begin{enumerate}
  \item Come up with a model for the conditional distribution of $y$
    given $\vec{x}$.
  \item Cast that conditional distribution as a member of the exponential family to determine what $\vec{\eta}$ is.
  \item Replace $\vec{\eta}_i$ with $\vec{\theta}_i^\top \vec{x}$.
  \item Come up with a procedure to maximize $\ell(\vec{\theta})$ for
    a training set.
  \end{enumerate}
    
\end{frame}


\begin{frame}{Generalized linear models}{Example: multinomial distribution}

  As an example, consider the generalization of the logistic
  regression problem to $k$ classes, i.e., ${\cal Y} = \{ 1, 2,
  \ldots, k \}$.

  \medskip

  The natural generalization of the Bernoulli distribution is the
  \alert{multinomial distribution} with parameters $\phi_1, \ldots \phi_{k-1}$.

  \medskip

  [We leave out the redundant $\phi_k = 1 - \sum_{i=1}^{k-1} \phi_i$.]

  \medskip

  To model the multinomial as a member of the exponential family,
  there is a rather involved derivation. See Bishop (2006) or Ng's
  lecture notes for details.
  
\end{frame}


\begin{frame}{Generalized linear models}{Example: multinomial distribution}

  The upshot: we can obtain $\eta_i = \log\frac{\phi_i}{\phi_k}$, which can be
  inverted to obtain
  \[ \phi_i = \frac{e^{\eta_i}}{\sum_{j=1}^k e^{\eta_j}}, \]
  which is called the \alert{softmax function} and is the multi-class
  generalization of the logistic sigmoid.

\end{frame}


\begin{frame}{Generalized linear models}{Example: multinomial distribution}

  Our prediction is thus
  \begin{eqnarray}
    p(y=i \mid \vec{x} ; \vec{\Theta}) & = & \phi_i \nonumber \\
    & = & \frac{e^{\eta_i}}{\sum_{j=1}^k e^{\eta_j}} \nonumber \\
    & = & \frac{e^{\vec{\theta}_i^\top\vec{x}}}{\sum_{j=1}^k e^{\vec{\theta}_j^\top\vec{x}}} \nonumber
  \end{eqnarray}

  \medskip

  \alert{Exercise}: play with the softmax function to see how it
  transforms log odds ratios into a probability mass distribution.

  \medskip

  See Bishop (2006) or Ng for the log likelihood function. As before,
  we can use gradient descent or Newton's method to find the optimal
  $\vec{\Theta}$.

\end{frame}


\begin{frame}{Generalized linear models}{Summary}

  To summarize the GLM approach:
  \begin{itemize}
    \item \textbf{Assumption 1}: the distribution $p(y \mid \vec{x} ;
      \vec{\theta})$ \alert{is a member of the exponential family}
      with natural parameter(s) $\vec{\eta}$.
    \item \textbf{Assumption 2}: \alert{our goal is to predict}, given
      an input $\vec{x}$, \alert{the expectation} $E[T(y) \mid \vec{x}
        ; \vec{\theta}]$. $T$ is a transformation of $y$ that comes
      from modeling $p(y \mid \vec{x} ; \vec{\theta})$ as a member of
      the exponential family.
    \item \textbf{Assumption 3}: the natural parameter(s) of the
      distribution $\vec{\eta}$ are \alert{linear in $\vec{x}$}, i.e.,
      $\eta_i = \vec{\theta}_i^\top\vec{x}$.
  \end{itemize}

  If you are willing to make these assumptions, you end up with a
  ``concave'' log likelihood function, which means that \alert{any
    local maximum is a global maximum}.

  \medskip

  A GLM should thus be a good first thing to try when you are faced
  with a machine learning problem you don't already have an algorithm
  for.
  
\end{frame}

%======================================================================
\section{Generative learning algorithms}
%======================================================================

\begin{frame}{Generative learning algorithms}{Generative vs. discriminative}

  So far, the methods we have tried attempt to learn
  $p(y \mid \vec{x})$ directly.

  \medskip

  Given an input $\vec{x}$, we try to map directly to the output $y$.

  \medskip
  
  Any algorithm that does this is called a \alert{discriminative}
  learning algorithm.

  \medskip

  Another class of algorithms instead tries to model $p(\vec{x} \mid
  y)$ and $p(y)$.

  \medskip

  Such methods are called \alert{generative} learning algorithms.
  
\end{frame}


\begin{frame}{Generative learning algorithms}{Using Bayes' rule}

  If we can build a model of $p(\vec{x} \mid y)$ and $p(y)$, then
  \alert{Bayes' rule} tells us that
  \[ p(y \mid \vec{x}) = \frac{p(\vec{x} \mid y) p(y)}{p(\vec{x})}. \]

  Why don't we care about $p(\vec{x})$?  Let's see...

  \medskip

  First, generative models are most often used for
  \alert{classification}, not regression.

  \medskip
  
  In classification, \alert{$y$ is discrete}, so we have
  \[ p(\vec{x}) = \sum_i p(\vec{x} \mid y=y_i) p(y=y_i) .\]

  \medskip

  If $y$ is continuous, we just have an integral instead of a sum.

  \medskip

  This shows that if we can model $p(\vec{x} \mid y)$ and $p(y)$,
  we can obtain $p(y \mid \vec{x})$ without explicitly calculating
  $p(\vec{x})$.

\end{frame}


\begin{frame}{Generative learning algorithms}{Using Bayes' rule}

  We could calculate $p(\vec{x})$ directly. But usually, we want to know
  \alert{which $y$ maximizes $p(y \mid \vec{x})$}.

  \medskip

  In this case, all we need to do is find
  \begin{eqnarray}
    y^* & = & \argmax_{y} p(y \mid \vec{x}) \nonumber \\
    & = & \argmax_{y} \frac{p(\vec{x} \mid y)p(y)}{p(\vec{x})} \nonumber \\
    & = & \argmax_y p(\vec{x} \mid y)p(y) \nonumber
  \end{eqnarray}

  \medskip

  So to perform classification in the generative approach, all we need
  is models for $p(\vec{x} \mid y)$ and $p(y)$.
  
\end{frame}


\begin{frame}{Generative learning algorithms}{Gaussian discriminant analysis}

  Let's consider the case where $p(\vec{x} \mid y)$ is a
  \alert{multivariate Gaussian}.

  \medskip

  A possible example would be ${\cal Y} = \{ \text{male},
  \text{female} \}$ and ${\cal X} = \Rset^2$ where the features are a
  person's height and weight.

\end{frame}


\begin{frame}{Generative learning algorithms}{Gaussian discriminant analysis}

  The $n$-dimensional multivariate Gaussian distribution has
  \begin{itemize}
  \item a \alert{mean vector} $\vec{\mu} \in \Rset^n$
  \item a \alert{covariance matrix} $\Sigma \in \Rset^{n\times
    n}$. We require $\Sigma \ge 0$ is symmetric and positive
    semidefinite.
  \end{itemize}

  \medskip

  The distribution is written ${\cal N}(\vec{\mu},\Sigma)$ and the
  density is
  \[ p(\vec{x} ; \vec{\mu}, \Sigma) = \frac{1}{(2\pi)^{n/2}|\Sigma|^{1/2}}
    e^{-\frac{1}{2}(\vec{x}-\vec{\mu})^\top \Sigma^{-1}(\vec{x}-\vec{\mu})} . \]
  
\end{frame}


\begin{frame}{Generative learning algorithms}{Gaussian discriminant analysis}

  If $\vec{X} \sim {\cal N}(\vec{\mu} ; \Sigma)$, then we can write
  
  \[ E[\vec{X}] = \int_{\vec{x}} \vec{x} p(\vec{x}; \vec{\mu},\Sigma)
  d\vec{x} = \vec{\mu} \]
  and
  \[ \cov(\vec{X}) = E[(\vec{X}-E[\vec{X}])(\vec{X}-E[\vec{X}])^\top] = \Sigma . \]

  \medskip

  See the handout on the multivariate Gaussian on the course Web site
  to get an intuition about interpreting the covariance matrix.
  
\end{frame}


\begin{frame}{Generative learning algorithms}{Gaussian discriminant analysis}

  Now, the GDA model for a 2-class problem:
  \begin{eqnarray}
    y & \sim & \textrm{Bernoulli}(\phi) \nonumber \\
    \vec{x} \mid y = 0 & \sim & {\cal N}(\vec{\mu}_0,\Sigma_0) \nonumber \\
    \vec{x} \mid y = 1 & \sim & {\cal N}(\vec{\mu}_1,\Sigma_1) \nonumber
  \end{eqnarray}

  \medskip

  We can then write the distributions:
  \begin{eqnarray}
    p(y) & = & \phi^y(1-\phi)^{1-y} \nonumber \\
    P(x \mid y=0) & = & \frac{1}{(2\pi)^{n/2}|\Sigma|^{1/2}} e^{-\frac{1}{2}(\vec{x}-\vec{\mu}_0)^\top \Sigma_0^{-1} (\vec{x}-\vec{\mu}_0)} \nonumber \\
      P(x \mid y=1) & = & \frac{1}{(2\pi)^{n/2}|\Sigma|^{1/2}} e^{-\frac{1}{2}(\vec{x}-\vec{\mu}_1)^\top \Sigma_1^{-1} (\vec{x}-\vec{\mu}_1)} \nonumber
  \end{eqnarray}

\end{frame}


\begin{frame}{Generative learning algorithms}{Gaussian discriminant analysis}

  Time to optimize!

  \begin{eqnarray}
    \ell(\phi,\vec{\mu}_0,\vec{\mu}_1,\Sigma_0,\Sigma_1) & = &
    \log \prod_{i=1}^m p(\vec{x}^{(i)},y^{(i)};\phi,\vec{\mu}_0,\vec{\mu}_1,
    \Sigma_0,\Sigma_1) \nonumber \\
    & = & \log \prod_{i=1}^m p(\vec{x}^{(i)} \mid y^{(i)};\vec{\mu}_0,\vec{\mu}_1,
    \Sigma_0,\Sigma_1) p(y^{(i)} ; \phi)\nonumber
  \end{eqnarray}

\end{frame}


\begin{frame}{Generative learning algorithms}{Gaussian discriminant analysis}

  It turns out that if we assume $\Sigma_0 = \Sigma_1 = \Sigma$, we obtain
  the maximum likelihood estimates
  \begin{eqnarray}
    \phi & = & \frac{1}{m}\sum_{i=1}^m1\{y^{(i)}=1\} \nonumber \\
    \vec{\mu}_0 & = & \frac{\sum_{i=1}^m\delta[y^{(i)}=0]\vec{x}^{(i)}}{\sum_{i=1}^m\delta[y^{(i)}=0]} \nonumber \\
    \vec{\mu}_1 & = & \frac{\sum_{i=1}^m\delta[y^{(i)}=1]\vec{x}^{(i)}}{\sum_{i=1}^m\delta[y^{(i)}=1]} \nonumber \\
    \Sigma & = & \frac{1}{m} \sum_{i=1}^m(\vec{x}^{(i)}-\vec{\mu}_{y^{(i)}})(\vec{x}^{(i)}-\vec{\mu}_{y^{(i)}})^\top
  \end{eqnarray}

  Note: I use $\delta[\psi]$ for a version of the \alert{Kronecker
    delta} function that is 1 if $\psi$ is true and 0 elsewhere.
  
\end{frame}


\begin{frame}{Generative learning algorithms}{Gaussian discriminant analysis}

  \begin{columns}

    \column{2.5in}
    
    \myfig{2.4in}{gda}{Ng, CS229 lecture note set \#2}
    
    \column{2.0in}
    
    The assumption
    \[ \Sigma_0 = \Sigma_1 = \Sigma \]
    turns out to be very convenient.

    \medskip
    
    \alert{Exercise:} Given the maximum likelihood estimates
    previously shown, try to find the set of points $\vec{x}$ for
    which
    \[ p(y=1 \mid \vec{x}) = 0.5. \]
    (It may help to note that this is where $p(\vec{x} \mid y=1 )p(y=1) =
    p(\vec{x}\mid y=0)p(y=0)$).

  \end{columns}
  
\end{frame}


\begin{frame}{Generative learning algorithms}{Gaussian discriminant analysis}

  The GDA model is related to the logistic regression model.

  \medskip

  Exercise: show that
  \[ p(y=1 \mid \vec{x} ; \phi, \Sigma, \mu_0, \mu_1) =
  \frac{1}{1+e^{-\vec{\theta}^\top \vec{x}}}, \]
  with $\vec{\theta}$ as a function of $\phi$, $\Sigma$, $\vec{\mu}_0$,
  $\vec{\mu}_1$.

  \medskip

  GDA and logistic regression have the same form but give different decision boundaries:
  \begin{itemize}
  \item GDA will be better (will require less training data to provide
    accurate predictions) if $p(\vec{x} \mid y)$ is in fact
    multivariate Gaussian or almost multivariate Gaussian.
  \item Logistic regression will probably be better if $p(\vec{x} \mid
    y)$ is definitely non-Gaussian or unknown.
  \end{itemize}
    
\end{frame}


\begin{frame}{Generative learning algorithms}{Naive Bayes}

  GDA was an example of a generative classification method for
  problems in which ${\cal X} = \Rset^n$.

  \medskip

  What if our features have \alert{discrete} values?

  \medskip

  Examples:
  \begin{itemize}
  \item Car buying behavior: ${\cal Y} = \{ \text{buy}, \neg\text{buy}
    \}$, ${\cal X} = \begin{bmatrix} \text{Size}, \text{Color},
      \text{Gender}, \text{Age} \end{bmatrix}$, $\text{Size} = \{
    \text{small}, \text{medium}, \text{large} \}$; $\text{Color} = \{
    \text{red}, \text{blue}, \text{black}, \text{white} \}$;
    $\text{Gender} = \{ \text{male}, \text{female} \}$; $\text{Age} =
    \{ \text{0-18}, \text{19-39}, \text{30-39}, \text{40-59},
    \text{60+} \}$.
    \item Email spam filter: ${\cal Y} = \{ \text{spam},
      \neg\text{spam} \}$. ${\cal X} = \{0,1\}^K$, with each variable
      $x_1,\ldots,x_k$ representing presence/absence of a particular
      word in a \alert{vocabulary}.
  \end{itemize}
  
\end{frame}


\begin{frame}{Generative learning algorithms}{Naive Bayes}

  Whatever $\cal Y$ and $\cal X$ are, a generative model
  needs a form for
  $p( y \mid \vec{x} )$.

  \medskip
  
  The simplest approach to these problems is to use the \alert{multinomial}
  distribution over the set of possible outcomes for $\vec{x}$.

  \medskip

  Exercise: How many outcomes for $\vec{x}$ are there for the car
  buying and spam filter examples?

  \medskip

  The full multinomial model will thus have $2L -1$ parameters
  for $p(\vec{x} \mid y)$, where $L$ is the number of outcomes for
  $\vec{x}$, and 1 parameter for $p(y)$, assuming $y$ is binary.

  \medskip

  Is this practical for the car buying example? For the spam example?

\end{frame}


\begin{frame}{Generative learning algorithms}{Naive Bayes}

  Naive Bayes attempts to reduce the number of parameters required
  using the (very strong but very useful) assumption that \alert{the
    features are conditionally independent given $y$}:
  \begin{eqnarray}
    p(\vec{x} \mid y) & = & p(x_1, x_2, \ldots, x_n \mid y) \nonumber \\
    & = & p(x_1 \mid y) p(x_2 \mid x_1, y) \cdots
    p(x_n \mid x_1, x_2, \ldots, x_{n-1}, y) \nonumber \\
    & \approx & \prod_{i=1}^n p(x_i \mid y) \nonumber
  \end{eqnarray}

  This model requires a set of parameters
  $\phi_{ij|y=1} = p(x_i = j \mid y = 1)$ and a set of parameters
  $\phi_{ij|y=0} = p(x_i = j \mid y = 0)$.

  \medskip

  Note that if the variables $x_i$ are binary, we only need two
  parameters $\phi_{i|y=1} = p(x_i = 1 \mid y = 1)$ and $\phi_{i|y=0}
  = p(x_i = 1 \mid y = 0)$.
  
\end{frame}


\begin{frame}{Generative learning algorithms}{Naive Bayes}

  For the case where $x_i$ are binary, we get the joint likelihood
  \[ {\cal L}(\phi_y, \phi_{i|y=0}, \phi_{i|y=1}) = \prod_{i=1}^m
  p(\vec{x}^{(i)},y^{(i)}) .\]

  \medskip

  Exercise: verify that maximizing ${\cal L}$ gives maximum likelihood
  estimates
  \begin{eqnarray}
    \phi_{j|y=1} & = &
    \frac{\sum_{i=1}^m \delta[x_j^{(i)} = 1 \wedge y^{(i)} = 1]}
         {\sum_{i=1}^m \delta[y^{(i)} = 1]} \nonumber \\
    \phi_{j|y=0} & = &
    \frac{\sum_{i=1}^m \delta[x_j^{(i)} = 1 \wedge y^{(i)} = 0]}
         {\sum_{i=1}^m \delta[y^{(i)} = 0]} \nonumber \\
    \phi_{y} & = &
    \frac{\sum_{i=1}^m \delta[y^{(i)} = 1]}
         {m} \nonumber
  \end{eqnarray}
  
\end{frame}


\begin{frame}{Generative learning algorithms}{Naive Bayes}

  If $n$ is large, there may be some features \alert{that do not appear in all
  combinations with every class}.

  \medskip

  Examples:
  \begin{itemize}
  \item The term ``viagra'' might occur only in spam emails.
  \item In a given period of time, there might not be any customers 18
    years or younger who bought a car.
  \end{itemize}

  \medskip

  Suppose we are then faced with an email $\vec{x}$ with the term
  ''viagra'' or a customer $\vec{x}$ whose age is 0-18.

  \medskip

  Do we predict $p(\text{spam} \mid \vec{x}) = 1$ and
  $p(\text{buy} \mid \vec{x}) = 0$ ?
  
\end{frame}


\begin{frame}{Generative learning algorithms}{Naive Bayes}

  \alert{Laplace smoothing} avoids 0-probability issues by adding
  one pseudo-example to the dataset:
  \begin{eqnarray}
    \phi_{j|y=1} & = &
    \frac{\sum_{i=1}^m \delta[x_j^{(i)} = 1 \wedge y^{(i)} = 1] + 1}
         {\sum_{i=1}^m \delta[y^{(i)} = 1] + 2} \nonumber \\
    \phi_{j|y=0} & = &
    \frac{\sum_{i=1}^m \delta[x_j^{(i)} = 1 \wedge y^{(i)} = 0] + 1}
         {\sum_{i=1}^m \delta[y^{(i)} = 0] + 2} \nonumber
  \end{eqnarray}

  This is for binary features (a multi-variate Bernoulli event model).

  \medskip

  See Ng's lecture notes for more information on the
  \alert{multinomial event model}.

\end{frame}


\begin{frame}{Generative learning algorithms}{Naive Bayes}

  So! Reminder:
  \begin{itemize}
  \item If you have continuous ${\cal X}$ and continuous ${\cal Y}$
    your first go-to model is \alert{linear regression}. Consider
    non-linear transformations of the inputs.
  \item If you have continuous ${\cal X}$ and discrete ${\cal Y}$ but
    don't know much about $p(\vec{x} \mid y)$, your first go-to model
    is \alert{logistic or softmax regression}, or you can come up with
    a \alert{new GLM} from scratch.
  \item If you have continuous ${\cal X}$ and discrete ${\cal Y}$ and
    know something about $p(\vec{x} \mid y)$, model the distribution
    accurately, as a Gaussian (\alert{GDA}) or build a \alert{new
      generative model} from scratch.
  \item If you have discrete ${\cal X}$ and discrete ${\cal Y}$ you
    should probably start with \alert{naive Bayes} then build up from
    there.
  \end{itemize}

  That's it for generative models. Next we'll return to discriminative
  models with kernel methods and SVMs.
  
\end{frame}

%======================================================================
\section{Support vector machines}
%======================================================================

\begin{frame}{Support vector machines}{Introduction}

  Thus far, we have studied machine learning models that are
  relatively easy to analyze and are \alert{optimal}, when the \alert{assumptions} they
  make are \alert{satisfied}.

  \medskip

  For example, GDA assumes the conditional distribution $p(\vec{x} \mid y)$ is
  a multivariate Gaussian.
  
  \medskip

  What happens when the assumptions are violated?

  \medskip

  Next we look at \alert{support vector machines (SVMs)}, which are more
  flexible and widely applicable than the method's we've looked at
  thus far.

  \medskip

  Although deep neural networks have received the most attention very
  recently, many still believe that the SVM is the best
  ``off-the-shelf'' supervised classifier.
  
\end{frame}


\begin{frame}{Support vector machines}{Introduction}

  SVMs are based on the idea of \alert{maximum margin} classification.

  \medskip

  Consider logistic regression, in which we model $p(y=1 \mid \vec{x}
  ; \vec{\theta})$ by $h_{\vec{\theta}}(\vec{x}) = g(\vec{\theta}^\top
  \vec{x})$.

  \medskip

  The logistic regression classification rule is
  \[ y^{pred}(\vec{x}) = \begin{cases} 1 & h_{\vec{\theta}}(\vec{x}) \ge 0.5 \\
    0 & \text{otherwise} \end{cases} \]

  Informally, our goal should be to find
  $\vec{\theta}$ such that
  $\vec{\theta}^\top \vec{x}^{(i)} \gg 0$ for all $i$ with $y^{(i)} = 1$ and
  $\vec{\theta}^\top \vec{x}^{(i)} \ll 0$ for all $i$ with $y^{(i)} = 0$.
  
\end{frame}


\begin{frame}{Support vector machines}{Decision boundaries}

  \begin{columns}

    \column{2.2in}
    
    \myfig{2in}{classes}{}

    \column{2.3in}

    Suppose the yellow points are training data from class 1 and the
    cyan points are training data from class 0.

    \medskip

    $\vec{\theta}^\top \vec{x} = 0$ is a \alert{separating hyperplane}
    or \alert{decision boundary} between the two classes.

    \medskip

    Point A is furthest from the decision boundary.  We should be
    confident to predict class 1 for point A.

    \medskip

    Point C is more ambiguous.

    \medskip

    This observation will lead to the principle of maximizing the margin.

  \end{columns}
  
\end{frame}


\begin{frame}{Support vector machines}{Model and notation}

  Next, we'll modify our model a little.

  \medskip

  Rather than classes 0 and 1, we'll make them -1 and +1.

  \medskip

  Rather than $\vec{\theta}^\top \vec{x}$ with implicit $x_0 = 1$, we
  write $\vec{w}^\top\vec{x} + b$ replacing $\theta_0$ by $b$, making
  the intercept term explicit.

  \medskip

  The classifier will be
  \[ h_{\vec{w},b}(\vec{x}) = g(\vec{w}^\top\vec{x}+b) .\]
  We use the perceptron rule
  \[ g(z) = \begin{cases} 1 & z \ge 0 \\ -1 & \text{otherwise} \end{cases} \]

\end{frame}


\begin{frame}{Support vector machines}{Margin}

  We have the structure of the model. What about the cost/objective function?

  \medskip

  First, we write the \alert{functional margin} of the parameters
  $\vec{w},b$ wih respect to training example $i$:
  \[ \hat{\gamma}^{(i)} = y^{(i)}(\vec{w}^\top\vec{x}+b) . \]
  Convince yourself that $\hat{\gamma}^{(i)}$ is
  \begin{itemize}
    \item 0 for training points on
      the decision boundary,
    \item negative for incorrectly classified points,
    \item positive for correctly classified points, and
    \item larger for correctly classified points further from the
      boundary.
  \end{itemize}

\end{frame}


\begin{frame}{Support vector machines}{Should we maximize the margin?}

  So $\hat{\gamma}^{(i)}$ seems like a good thing to maximize.

  \medskip

  However, there is one problem with $\hat{\gamma}^{(i)}$ as defined so far.

  \medskip

  \alert{Scaling} $\vec{w}$ and $b$ by an arbitrary factor $\alpha$
  does not change the decision boundary but does scale the resulting
  margin by $\alpha$.

  \medskip

  That means $\hat{\gamma}^{(i)}$ could be made artificially large by
  scaling with a large $\alpha$.

  \medskip

  We will avoid this by constraining $\vec{w}$ to be length 1. Instead
  of $(\vec{w},b)$, we will use $(\vec{w}/\|\vec{w}\|,b/\|\vec{w}\|)$
  when calculating the margin.

  \medskip

  This gives us the \alert{geometric margin}
  \[ \gamma^{(i)} = y^{(i)}\left( \left(\frac{\vec{w}}{\|\vec{w}\|}\right)^\top\vec{x}^{(i)} + \frac{b}{\|\vec{w}\|} \right) \]

\end{frame}


\begin{frame}{Support vector machines}{Maximize the minimum margin}

  What should our goal actually be?

  \medskip

  We could think of logistic regression as attempting to maximize a kind
  of average margin.

  \medskip

  However, SVMs attempt to maximize the \alert{minimum} margin.

  \medskip
  
  So we will define the functional margin with respect to an entire training set $S = \{ (\vec{x}^{(i)},y^{(i)}) \}_{i \in 1..m}$ as
  \[ \gamma = \min_{i\in 1..m} \gamma^{(i)}, \]
  i.e., the smallest of the margins of the individual
  training examples.

\end{frame}


\begin{frame}{Support vector machines}{The optimization problem}

  OK! Now we know that we'd like to find $\vec{w},b$ according to
  \[
  \begin{array}{rl}
    \max_{\gamma,\vec{w},b} & \gamma \\ \text{subject to} & y^{(i)}(\vec{w}^\top\vec{x}^{(i)} + b) \ge \gamma, i \in 1..m \\ & \| \vec{w} \| = 1.
  \end{array}
  \]
  Only problem: the constraint $\| \vec{w} \| = 1$ is not linear,
  as would be required by most optimizers.

  \medskip

  Think of the constraint as saying we must find a point on the unit
  sphere that maximizes $\gamma$. Tricky!
  
\end{frame}


\begin{frame}{Support vector machines}{The optimization problem}

  First step to improve the formulation of
  the optimization problem: \alert{move the constraint
  into the objective}:
  \[
  \begin{array}{rl}
    \max_{\gamma,\vec{w},b} & \frac{\hat{\gamma}}{\| \vec{w} \|} \\
    \text{subject to} & y^{(i)}(\vec{w}^\top\vec{x}^{(i)} + b) \ge \hat{\gamma}, i \in 1..m .
  \end{array}
  \]
  The good: the constraints are now all \alert{linear}. That is good
  for standard optimizers.

  \medskip

  New problem: now the objective function $\vec{\hat{\gamma}}/{\|\vec{w}\|}$
  is itself non-convex.

  \medskip

  Can we fix it?  Well, first let's note that the constraint $\|\vec{w}\|=1$
  is fairly arbitrary. Actually, any constraint on $\vec{w}$ that prevents
  the optimizer from scaling $\vec{w}$ to get a bigger margin is enough.

  \medskip

  Other constraints such as $w_1 = 10$ or $\sqrt{w_1^2 + w_2^2} = 3$
  would be equally good.
  
\end{frame}


\begin{frame}{Support vector machines}{The optimization problem}

  The particular constraint on $\vec{w},b$ that we will use is a
  little weird:
  \[ \hat{\gamma}=1 .\]
  That is, we will scale $\vec{w},b$ so that the distance of the training
  point(s) closest to the hyperplane $\vec{w}^\top \vec{x} + b = 0$ is 1.

  \medskip

  Replacing $\hat{\gamma}$ with 1 in our optimization problem, we can
  obtain
  \[
  \begin{array}{rl}
    \min_{\gamma,\vec{w},b} & \frac{1}{2} \| \vec{w} \|^2 \\
    \text{subject to} & y^{(i)}(\vec{w}^\top\vec{x}^{(i)} + b) \ge 1, i \in 1..m .
  \end{array}
  \]

  Convince yourself that this is a convex optimization problem with
  linear constraints that ``rule out'' parts of the parameter space as
  candidate solutions.
  
\end{frame}


\begin{frame}{Support vector machines}{Quadratic programming problem}

  So! If we want to find the \alert{optimal margin classifier} for a linearly
  separable data set $\{(y^{(i)},\vec{x}^{(i)})\}_{i \in 1..m}$, all we need to
  do is use a quadratic programming solver to find $\vec{w},b$ satisfying
  \[
  \begin{array}{rl}
    \min_{\gamma,\vec{w},b} & \frac{1}{2} \| \vec{w} \|^2 \\
    \text{subject to} & y^{(i)}(\vec{w}^\top\vec{x}^{(i)} + b) \ge 1, i \in 1..m .
  \end{array}
  \]

  See the formalization of the 
  general quadratic programming problem on the
  next page...

\end{frame}


\begin{frame}{Support vector machines}{Quadratic programming problem}

  \begin{block}{The (general) \alert{quadratic programming} problem}
    Given
    \begin{itemize}
    \item $\vec{c} \in \Rset^n$
    \item $\mat{Q} \in \Rset^{n\times n}$
    \item $\mat{A} \in \Rset^{m\times n}$
    \item $\vec{b} \in \Rset^m$
    \end{itemize}
    find \[ \begin{array}{rl}
      \vec{x}^* \in \Rset^n = & \argmin_{\vec{x}} \frac{1}{2} \vec{x}^\top \mat{Q} \vec{x} + \vec{c}^\top \vec{x} \\
      & \text{subject to} \;\;\; \mat{A}\vec{x}\le \vec{b}
    \end{array} \]
  \end{block}

  [Check that our problem can be cast as a QP problem.]
  
\end{frame}


\begin{frame}{Support vector machines}{Quadratic programming problem}

  There are many standard QP solvers.

  \medskip

  For example, in Python, try \texttt{cvxopt}:
  \begin{itemize}
  \item \texttt{sudo pip3 install cvxopt}
  \end{itemize}
  See the sample Jupyter notebook on the course home page.

\end{frame}


\begin{frame}{Support vector machines}{The dual optmization problem}

  Turns out that although we could stop here, QP is \alert{not efficient in
    high dimensional spaces}.

  \medskip

  We will see later that we want to build SVMs in high dimensional
  feature spaces in order to get interesting non-linear classification
  boundaries in the original attribute space.

  \medskip

  So instead of throwing our direct
  SVM problem at a QP solver, we will reformulate
  the problem in a way that it can be solved efficiently in high dimensional
  spaces.

  \medskip

  This requires a discussion of Lagrangian optimization.

\end{frame}


\begin{frame}{Support vector machines}{Lagrangian optmization}

  Let's consider a 2-dimensional example where we want to maxmimize
  $f(x,y)$ subject to $g(x,y)=0$:

  \myfig{3.2in}{lagrange}{\url{https://en.wikipedia.org/wiki/Lagrange_multiplier}}

\end{frame}


\begin{frame}{Support vector machines}{Lagrangian optmization}

  We introduce a new variable $\lambda$ called a \alert{Lagrange multiplier}
  and consider the \alert{Lagrange function} or \alert{Lagrangian}
  \[ {\cal L}(x,y,\lambda) = f(x,y) - \lambda g(x,y). \]
  It turns out that if $(x^*,y^*)$ is a maximum of $f(x,y)$ satisfying
  $g(x^*,y^*)=0$, then there exists a $\lambda^*$ such that
  $(x^*,y^*,\lambda^*)$ is a \alert{stationary point} of ${\cal
    L}(x,y,\lambda)$.

  \medskip

  That just means $(x^*,y^*,\lambda^*)$ will satisfy
  \[ \frac{\partial {\cal L}}{\partial x} = \frac{\partial {\cal L}}{\partial y} = \frac{\partial {\cal L}}{\partial \lambda} = 0. \]
  
\end{frame}


\begin{frame}{Support vector machines}{Lagrangian optmization}

  OK! So if we want to find the constrained maximum of $f(x,y)$ subject
  to $g(x,y)=0$, we just need
  to
  \begin{itemize}
  \item Write the Lagrangian.
  \item Set the derivatives of the Lagrangian to 0.
  \item Solve the resulting system
    of equations.
  \item Check whether the solution is a maximum or not.
  \end{itemize}

  \medskip

  A constrained minimum would be found in the same way, with a check
  for a minimum (see next page).

\end{frame}


\begin{frame}{Support vector machines}{Lagrangian optmization}

  \begin{block}{Hessian analysis for
      an \alert{unconstrained} optimization problem} A \alert{minimum}
    of $f(\vec{w})$ will have a Hessian $\mat{H}_f$ with all
    \alert{positive} eigenvalues at $\vec{w}^*$, and a \alert{maximum}
    will have a Hessian with all \alert{negative} eigenvalues.
  \end{block}
    
  \begin{block}{Hessian analysis for
      a \alert{constrained} optimization problem}
    For constrained optimization problems, critical points of the
    Lagrangian are \alert{always saddle points} ($\mat{H}_{\cal L}$
    will have a mix of negative and positive eigenvalues).

    \medskip

    For $m$ constraints and $n$ total variables including the Lagrange
    multipliers, we examine the determinants of the $2m+1$th to $n$th principal minors of
    $\mat{H}_{\cal L}$. If the critical point is a minimum, all of the minors' determinants will have a sign of $(-1)^m$. A maximum will have alternating
    signs of the minors' determinants!
  \end{block}
    
\end{frame}


\begin{frame}{Support vector machines}{Lagrangian optmization}

  It would be worthwhile to do a simple example at this point to make
  sure we've got it.

  \medskip

  Suppose we have $f(\vec{w}) = \|\vec{w}\|^2$ and $g(\vec{w}) =
  w_1 + w_2 - 4$.

  \medskip

  Go ahead and draw a visualization of $f(\cdot)$ and $g(\cdot)$. You
  should be able to determine by inspection the optimal $\vec{w}^*$
  (a minimum).

  \medskip

  Then try to find $\vec{w}^*$ using the Lagrangian and verify it.
  
\end{frame}


\begin{frame}{Support vector machines}{Generalized Lagrangian optmization}

  Next, we generalize the Lagrangian notion to an arbitrary parameter
  vector and the case where we have \alert{inequality} constraints in
  addition to the equality constraints.

  \medskip

  Following Ng's lecture notes, set 3, we aim to solve the
  optimization problem
  \[ \begin{array}{rl}
    \min_{\vec{w}} & f(\vec{w}) \\
    \text{s.t.}  & g_i(\vec{w}) \le 0, i \in 1..k \\
    & h_i(\vec{w}) = 0, i \in 1..l
    \end{array} \]

  \medskip

  We write the \alert{generalized Lagrangian}
  \[ {\cal L}(\vec{w},\vec{\alpha},\vec{\beta}) =
  f(\vec{w}) + \sum_{i=1}^k \alpha_i g_i(\vec{w}) + \sum_{i=1}^l \beta_i h_i(\vec{w}) \]

\end{frame}


\begin{frame}{Support vector machines}{Lagrangian duality}

  Turns out there are a couple ways to solve the optimization problem using
  the generalized Lagrangian.

  \medskip

  First: the \alert{primal} problem:
  \begin{eqnarray}
    \theta_{\cal P}(\vec{w}) & = & \max_{\vec{\alpha},\vec{\beta}:\alpha_i \ge 0} {\cal L}(\vec{w},\vec{\alpha},\vec{\beta}) \nonumber\\
    & = & \max_{\vec{\alpha},\vec{\beta}:\alpha_i \ge 0} f(\vec{w}) + \sum_{i=1}^k \alpha_i g_i(\vec{w}) + \sum_{i=1}^l \beta_i h_i(\vec{w}) \nonumber
    \end{eqnarray}

  \medskip

  Convince yourself that if $\vec{w}$ violates any of our constraints,
  then $\theta_{\cal P}(\vec{w}) = \infty$, and if $\vec{w}$ satisfies
  all the constraints, then $\theta_{\cal P}(\vec{w}) = f(\vec{w})$.
  
\end{frame}


\begin{frame}{Support vector machines}{Lagrangian duality}

  So! If you convinced yourself of the fact on the previous page, we can now
  say that the $\vec{w}$ satisfying
  \[ \min_{\vec{w}} \theta_{\cal P}(\vec{w}) = \min_{\vec{w}} \max_{\vec{\alpha},\vec{\beta}:\alpha_i \ge 0} {\cal L}(\vec{w},\vec{\alpha},\vec{\beta}) \]
  solves the original optimization problem.

  \medskip

 We call $p^* = \min_{\vec{w}}\theta_{\cal P}(\vec{w})$ the
 \alert{value of the primal problem}.

  \medskip

  It turns out the primal problem is sometimes not as easy to solve as
  another problem, the \alert{dual} problem, that has the same solution
  under certain conditions.
    
\end{frame}


\begin{frame}{Support vector machines}{Lagrangian duality}

  So to get to a problem that might be easier to solve,
  consider the \alert{dual} optimization problem in which we let
  \[ \theta_{\cal D}(\vec{\alpha},\vec{\beta}) = \min_{\vec{w}}{\cal L}(\vec{w},\vec{\alpha},\vec{\beta}). \]

  All we've done is swapped the order of the minimization and maximization.

  \medskip

  The \alert{dual optimization problem} is
  \[ \max_{\vec{\alpha},\vec{\beta}:\alpha_i \ge 0} \theta_{\cal D}(\vec{\alpha},\vec{\beta}) = \max_{\vec{\alpha},\vec{\beta}:\alpha_i \ge 0} \min_{\vec{w}} {\cal L}(\vec{w},\vec{\alpha},\vec{\beta}). \]
  
  The \alert{value of the dual problem's objective} is $d^*
  =\max_{\vec{\alpha},\vec{\beta}:\alpha_i \ge 0} \theta_{\cal
    D}(\vec{\alpha},\vec{\beta})$

\end{frame}


\begin{frame}{Support vector machines}{Lagrangian duality}

  It is always true that
  \[ d^* = \max_{\vec{\alpha},\vec{\beta}:\alpha_i \ge 0}\min_{\vec{w}}{\cal L}(\vec{w},\vec{\alpha},\vec{\beta}) \le
  \min_{\vec{w}} \max_{\vec{\alpha},\vec{\beta}:\alpha_i \ge 0} {\cal
    L}(\vec{w},\vec{\alpha},\vec{\beta}) = p^* .\]

  Under certain circumstances, it will also be the case that
  \[ d^* = p^* ,\]
  which would mean we can solve the dual problem instead of the primal
  problem if we so choose.

\end{frame}


\begin{frame}{Support vector machines}{Lagrangian duality}

  What are the conditions under which $p^*=d^*$? It turns out that if
  \begin{itemize}
  \item $f$ is convex\footnote{Loosely speaking, if we consider
    the points $\vec{w}$ on a straight line
    between $\vec{w}_1$ and $\vec{w}_2$, $f(\vec{w})$ is always below
    the line.}
  \item The $g_i$'s are convex
  \item The $h_i$'s are affine
  \item The $g_i$'s are strictly feasible (there is a $\vec{w}$ such that $g_i(\vec{w})<0$ for all $i$)
  \end{itemize}

  \medskip

  If these conditions hold, ... (see next page)

\end{frame}


\begin{frame}{Support vector machines}{Lagrangian duality}

  If the conditions on the previous page are met,
  there must exist $\vec{w}^*,\vec{\alpha}^*,\vec{\beta}^*$ so
  that
  \begin{itemize}
  \item  $\vec{w}^*$ is the solution to the primal problem,
  \item $\vec{\alpha}^*,\vec{\beta}^*$ are the solution to the dual problem,
  \item $p^*=d^*={\cal L}(\vec{w}^*,\vec{\alpha}^*,\vec{\beta}^*)$, and
  \item $\vec{w}^*,\vec{\alpha}^*,\vec{\beta}^*$ satisfy the
    \alert{Karush-Kuhn-Tucker (KKT) conditions}\footnote{First discovered
    by Karush in a 1939 Master's thesis, then rediscovered by Kuhn and
    Tucker in 1951.} (next page)!
  \end{itemize}

  The KKT conditions will allow us to easily find the optimal
  $\vec{w}^*,\vec{\alpha}^*,\vec{\beta}^*$.
  
\end{frame}


\begin{frame}{Support vector machines}{Lagrangian duality}

  If the assumptions are met, then 
  any $\vec{w}^*,\vec{\alpha}^*,\vec{\beta}^*$ satisfying the KKT
  conditions are solutions to both the primal and dual problem. The conditions:
  \begin{eqnarray}
    \frac{\partial}{\partial w_i}{\cal L}(\vec{w}^*,\vec{\alpha}^*,\vec{\beta}^*) & = & 0, i \in 1..n \nonumber \\
    \frac{\partial}{\partial \beta_i}{\cal L}(\vec{w}^*,\vec{\alpha}^*,\vec{\beta}^*) & = & 0, i \in 1..l \nonumber \\
    \alpha_i^*g_i(\vec{w}^*) & = & 0, i \in 1..k \nonumber \\
    g_i(\vec{w}^*) & \le & 0, i \in 1..k \nonumber \\
    \alpha_i^* & \ge & 0, i \in 1..k \nonumber
  \end{eqnarray}

  Pay attention to the 3rd condition. This will enable us to show that
  the SVM has only a few support vectors.

\end{frame}


\begin{frame}{Support vector machines}{Dual optimization problem for SVMs}

  OK, now we know how to write the primal and dual optimization problems
  for SVMs.

  \medskip

  Previously we had
  \[
  \begin{array}{rl}
    \min_{\gamma,\vec{w},b} & \frac{1}{2} \| \vec{w} \|^2 \\
    \text{subject to} & y^{(i)}(\vec{w}^\top\vec{x}^{(i)} + b) \ge 1, i \in 1..m .
  \end{array}
  \]  
  To put this into the dual Lagrangian framework, we write the constraints
  as
  \[ g_i(\vec{w}) = -y^{(i)}(\vec{w}^\top \vec{x}^{(i)} + b) + 1 \le 0 \]
  i.e., one constraint for each training example.

  \medskip

  From the KKT conditions, we will have $\alpha_i > 0$ only for
  training examples with functional margin exactly 1, where
  $g_i(\vec{w}) = 0$.

\end{frame}


\begin{frame}{Support vector machines}{Dual optimization problem for SVMs}

  \myfig{2.5in}{svs}{Ng, CS229 lecture notes set 3}

  \medskip

  The points on the dashed lines are the ones for which $g_i(\vec{w})=0$ and
  $\alpha_i>0$. They are called the \alert{support vectors}.
  
\end{frame}


\begin{frame}{Support vector machines}{Dual optimization problem for SVMs}

  Next let's write the generalized Lagrangian for our optimization problem:
  \[ {\cal L}(\vec{w},b,\vec{\alpha}) = \frac{1}{2}\|\vec{w}\|^2
  - \sum_{i=1}^m \alpha_i \left[ y^{(i)}(\vec{w}^\top\vec{x}^{(i)} +b)-1\right] \]

  There are only $\alpha_i$'s not $\beta_i$'s because the problem only
  has inequality constraints, not equality constraints.
  
\end{frame}


\begin{frame}{Support vector machines}{Dual optimization problem for SVMs}

  For the dual optimization problem, we first minimize ${\cal
    L}(\vec{w},b,\vec{\alpha})$ with respect to $\vec{w}$ and $b$, for
  \alert{fixed} $\vec{\alpha}$.

  \medskip

  To find the $\vec{w}$ and $b$, we take the partials and set to 0:
  \[ \nabla_{\vec{w}}{\cal L}(\vec{w},b,\vec{\alpha}) =
  \vec{w} - \sum_{i=1}^m\alpha_i y^{(i)}\vec{x}^{(i)} = \vec{0} \]
  from which we can obtain
  \[ \vec{w} = \sum_{i=1}^m \alpha_i y^{(i)} \vec{x}^{(i)} .\]
  For $b$ we obtain the equation
  \[ \frac{\partial}{\partial b}{\cal L}(\vec{w},b,\vec{\alpha}) =
  \sum_{i=1}^m\alpha_i y^{(i)} = 0 \]
  This doesn't tell us anything about $b$, note, but we'll solve that problem
  shortly.

\end{frame}


\begin{frame}{Support vector machines}{Dual optimization problem for SVMs}

  Now that we have a definition of $\vec{w}$ (previous slide), we plug
  into the Langrangian, obtaining
  \[ {\cal L}(\vec{w},b,\vec{\alpha}) = \sum_{i=1}^m\alpha_i - \frac{1}{2}\sum_{i,j=1}^m y^{(i)}y^{(j)}\alpha_i\alpha_j(\vec{x}^{(i)})^\top\vec{x}^{(j)}- b\sum_{i=1}^m\alpha_i y^{(i)}. \]
  Using the result from $\partial {\cal L}/\partial b$, we know the
  last term is $\vec{0}$, reducing us to
  \[ {\cal L}(\vec{w},b,\vec{\alpha}) = \sum_{i=1}^m\alpha_i - \frac{1}{2}\sum_{i,j=1}^m y^{(i)}y^{(j)}\alpha_i\alpha_j(\vec{x}^{(i)})^\top\vec{x}^{(j)}. \]
  
\end{frame}


\begin{frame}{Support vector machines}{Dual optimization problem for SVMs}

  Next, for the dual optimization problem, now that we've written
  down the equivalent of 
  $\theta_{\cal D}(\vec{\alpha},\vec{\beta})$, our job is to now maximize
  this expression w.r.t.\ $\vec{\alpha}$.
 
  \medskip

  We'll write $\theta_{\cal D}$ as $W$ for the specific case of the SVM:
  \[ \begin{array}{rl}
    \max_{\vec{\alpha}} & W(\vec{\alpha}) = \sum_{i=1}^m\alpha_i - \frac{1}{2}\sum_{i,j=1}^m y^{(i)}y^{(j)}\alpha_i\alpha_j\left< \vec{x}^{(i)},\vec{x}^{(j)}\right> \\
    \text{such that} & \alpha_i \ge 0, i \in 1..m \\
    & \sum_{i=1}^m\alpha_iy^{(i)} = 0
  \end{array} \]
  where $\left<\cdot,\cdot\right>$ is the inner product.

  \medskip
  
  If the assumptions for $p^*=d^*$ hold (verify that they do), then we
  can use the KKT conditions to solve this dual problem instead of
  solving the primal problem.
  
\end{frame}


\begin{frame}{Support vector machines}{Dual optimization problem for SVMs}

  Once we solve the dual problem for $\vec{w}$ to obtain $\vec{w}^*$,
  we have the \alert{orientation} of our maximum margin hyperplane,
  and all that remains is to find $b$, the bias or signed distance of
  the hyperplane to the origin.

  \medskip

  \[ b^* = -\frac{\max_{i:y^{(i)}=-1}\vec{w}^{*\top}\vec{x}^{(i)} + \min_{i:y^{(i)}=1}\vec{w}^{*\top}\vec{x}^{(i)}}{2} \]

  \medskip

  You should be able to verify this easily.

  \medskip

  Note that another way to calculate $b$ (not knowing $\vec{w}$) is
  \[ b = \frac{1}{N_{\cal S}}\sum_{i\in {\cal S}} \left(
         y^{(i)} - \sum_{j\in {\cal S}} \alpha_j y^{(j)} \left< \vec{x}^{(i)},
         \vec{x}^{(j)} \right> \right), \]
  where ${\cal S}$ is the set of indexes of the support vectors $\{i \in 1..m \mid
  \alpha_i > 0\}$ and $N_{\cal S}$ is the size of that set.

\end{frame}


\begin{frame}{Support vector machines}{Prediction using SVM}

  Once we have the optimal $\vec{w}$ and $b$, we would make the
  \alert{prediction}
  \[ h_{\vec{w},b}(\vec{x}) = \begin{cases} +1 & \vec{w}^\top \vec{x}+b \ge 0 \\
    -1 & \text{otherwise} \end{cases} \]
  However, it is important to note that
  \begin{eqnarray}
    \vec{w}^\top \vec{x}+b & = & \left( \sum_{i=1}^m\alpha_i y^{(i)}\vec{x}^{(i)} \right)^\top \vec{x} + b \nonumber \\
    & = & \sum_{i=1}^m\alpha_i y^{(i)}\left< \vec{x}^{(i)}, \vec{x}\right> + b .\nonumber
  \end{eqnarray}
  which shows that our prediction for $\vec{x}$ is a \alert{weighted
    summation of inner products} with the \alert{support vectors} (the
  $\vec{x}^{(i)}$ for which $\alpha_i \ne 0$).
  
\end{frame}


\begin{frame}{Support vector machines}{Coming next}

  Well, now we've understood very well the optimization problem for
  support vector machines in the linearly separable case.

  \medskip

  As of yet, we haven't discussed:
  \begin{itemize}
  \item The case where the training data are \alert{not
    linearly separable}.
  \item How we can turn the linear SVM into a \alert{nonlinear} classifier.
  \item An \alert{efficient optimization algorithm} for the $\alpha_i$'s
  \end{itemize}

  These issues are coming next!

  \medskip

  We'll introduce \alert{kernels} as a way to obtain nonlinear
  transformations of the input data then perform linear classification
  in the transformed space.

  \medskip

  We'll introduce \alert{regularization} to allow for non-separable
  data and have some resilience to outliers.

  \medskip

  We'll look at how a new optimization algorithm, \alert{coordinate
    ascent}, can be adapted to the case of our dual optimization
  problem.

  \medskip
  
  ``Raise your hand if this makes sense!''

\end{frame}


\begin{frame}{Support vector machines}{Kernels: feature mapping}

  When we learned linear regression, we found that it was a simple
  matter to generate new features that were nonlinear transforms of
  the original attributes.

  \medskip

  Let's make some terminology:
  \begin{itemize}
  \item Let's call the original inputs for $\vec{x}$ in the training
    set the input \alert{attributes} for the problem.
  \item We'll call the newly generated values the input \alert{features}.
  \item Let $\vec{\phi}$ denote the \alert{feature mapping} that maps attributes
    to features.
  \end{itemize}

  \medskip

  Example: if we mapped an original scalar feature $x$ (such as height)
  to $x$, $x^2$, and $x^3$ to obtain a cubic hypothesis, we would write
  the mapping
  \[ \vec{\phi}(x) = \begin{bmatrix} x \\ x^2 \\ x^3 \end{bmatrix} .\]

\end{frame}


\begin{frame}{Support vector machines}{Kernels}

  To allow for mapping in the SVM optimization algorithm, we simply
  replace $\vec{x}$ with $\vec{\phi}(\vec{x})$.

  \medskip

  Where we have inner products $\left< \vec{x}, \vec{z} \right>$, we
  will have instead $\left< \vec{\phi}(\vec{x}),\vec{\phi}(\vec{z})
  \right>$

  \medskip

  New term: given a feature mapping $\vec{\phi}$, the corresponding
  \alert{kernel} is
  \[ K(\vec{x},\vec{z}) = \vec{\phi}(\vec{x})^\top\vec{\phi}(\vec{z}) .\]

  \medskip

  Now we can replace $\left< \vec{x}, \vec{z} \right>$ with
  $K(\vec{x},\vec{z})$.

\end{frame}


\begin{frame}{Support vector machines}{Kernels: prediction using kernels}

  Note: thus far we haven't done much!

  \medskip

  We have just added a mapping function and rewritten the SVM inference
  algorithm in terms of a kernel function.

  \medskip

  Instead of computing
  \[ \sum_{i=1}^m\alpha_i y^{(i)}\left< \vec{x}^{(i)}, \vec{x}\right> + b ,\]
  then predicting based on the sign of the result, we have rewritten the
  computation as
  \[ \sum_{i=1}^m\alpha_i y^{(i)} K(\vec{x}^{(i)}, \vec{x}) + b .\]
  
\end{frame}


\begin{frame}{Support vector machines}{Kernels: efficiency}

  What is the benefit of rewriting our inference rule in terms of kernels?

  \medskip

  It turns out that sometimes computing the kernel is \alert{more
    efficient} than transforming the input attributes into feature
  space then computing a dot product.

  \medskip

  Ng's example: consider the kernel
  \[ K(\vec{x},\vec{z}) = (\vec{x}^\top\vec{z})^2. \]
  What $\vec{\phi}$ does this correspond to?

\end{frame}


\begin{frame}{Support vector machines}{Kernels: efficiency}

  \begin{eqnarray}
    K(\vec{x},\vec{z}) & = & (\vec{x}^\top\vec{z})^2 \nonumber \\
    & = & \left( \sum_{i=1}^n x_iz_i \right)
    \left( \sum_{j=1}^n x_j z_j \right) \nonumber \\
    & = & \sum_{i=1}^n\sum_{j=1}^n x_i x_j z_i z_j \nonumber \\
    & = & \sum_{i,j=1}^n (x_i x_j)(z_i z_j) \nonumber
  \end{eqnarray}
  which you can verify corresponds to
  $\vec{\phi}(\vec{x})^\top\vec{\phi}(\vec{z})$ with (using $n=3$ for example)
  \[ \vec{\phi}(\vec{x}) = \begin{bmatrix} x_1x_1 & x_1x_2 & x_1x_3 & x_2x_1 & x_2x_2 & x_2x_3 & x_3x_1 & x_3x_2 & x_3x_3 \end{bmatrix}^\top .\]
      
\end{frame}


\begin{frame}{Support vector machines}{Kernels: efficiency}

  In the example, computing $\vec{\phi}(\vec{x})$ takes $O(n^2)$ time,
  but computing $K(\vec{x},\vec{z})$ takes $O(n)$ time!

  \medskip

  So we see that kernels can be more efficient than dot products in feature
  space.

\end{frame}


\begin{frame}{Support vector machines}{Kernels: interpretation}

  Kernels are efficient. Good. Then how can we think about what kernel
  to use for a particular problem?

  \medskip

  Consider that a dot product to a certain extent measures how similar
  two vectors are:
  \begin{itemize}
  \item The dot product is 0 for orthogonal vectors
  \item The dot product grows as we rotate the vectors toward each other.
  \end{itemize}

  \medskip

  We can use this intuition to generate ideas for different kinds of kernels.

\end{frame}


\begin{frame}{Support vector machines}{Kernels: interpretation}

  For example, the most popular kernel besides the linear kernel is the
  \alert{Gaussian kernel} or \alert{radial basis function (RBF) kernel}
  \[ K(\vec{x},\vec{z}) = e^{-\frac{\|\vec{x}-\vec{z}\|^2}{2\sigma^2}} \]
  which is 1 when $\vec{x}=\vec{z}$ and decreases as they move apart.

  \medskip

  It turns out that this choice of $K(\cdot,\cdot)$ is a valid kernel,
  although it corresponds to a $\vec{\phi}$ transforming $\vec{x}$ into
  \alert{an infinite dimensional feature space}!!
  
\end{frame}


\begin{frame}{Support vector machines}{Kernels: validity}

  OK so we have two ways to design kernels:
  \begin{itemize}
  \item Come up with a $\vec{\phi}$ we think useful, and compute $K(\cdot,\cdot)$ in the feature space directly.
    \item Come up with a $K(\cdot,\cdot)$ we think useful, verify that
      it is a valid kernel, and just use it without ever computing
      $\vec{\phi}(\vec{x})$ at all.
  \end{itemize}

  \medskip

  So we need to know what makes a kernel valid.

  \medskip

  \textbf{Mercer's theorem} (roughly): $K(\cdot,\cdot)$ is a valid
  kernel iff the \alert{kernel matrix} or \alert{Gram matrix}
  \[ \mat{K} = \begin{bmatrix} K(\vec{x}^{(1)},\vec{x}^{(1)}) &
    K(\vec{x}^{(1)},\vec{x}^{(2)}) & \cdots \\
    K(\vec{x}^{(2)},\vec{x}^{(1)}) & K(\vec{x}^{(2)},\vec{x}^{(2)}) & \cdots \\
    \vdots & \vdots & \ddots \end{bmatrix} \]
  is symmetric positive semidefinite for any training set $(\vec{x}^{(1)},
  \ldots,\vec{x}^{(m)})$.

\end{frame}


\begin{frame}{Support vector machines}{Kernels: validity}

  See Hastie et al., Bishop, or Ng for (a lot) more detail on valid
  kernels, proofs of positive semidefiniteness of the kernel matrix, and
  so on!

  \medskip

  If we're doing research on kernel methods, we'll need those details.

  \medskip

  Also, we can apply the \alert{kernel trick} to any machine learning algorithm, not just SVMs, to get a ``kernelized'' version of that algorithm.

  \medskip

  So, like the GLM trick, the kernel trick is a kind of generator of
  new machine learning algorithms.
  
  \medskip

  If we are primarily a user, however, we can just try the various kernel
  options provided by a SVM library.
  
\end{frame}


\begin{frame}{Support vector machines}{Non-separable data}

  Recall what allowed us to formulate the SVM maximum margin problem
  as a QP problem: \alert{linear separability made the constraints satisfiable}.

  \medskip

  Now suppose the training data are \alert{not} linearly separable
  and/or contain \alert{outliers}.

  \medskip

  We can extend our objective function introducing \alert{slack variables}:
  \[ \begin{array}{rl}
    \min_{\vec{\gamma},\vec{w},b,\vec{\xi},\vec{r}} & \frac{1}{2}\|\vec{w}\|^2 + C\sum_{i=1}^m \xi_i \\
    \text{such that} & y^{(i)}(\vec{w}^\top \vec{x}^{(i)}+b) \ge 1-\xi_i, i \in 1..m \\
    & \xi_i \ge 0, i \in 1..m
  \end{array} \]

  This is still a QP problem (now we have a linear term in the
  objective function and a modified set of linear inequality
  constraints).

  \medskip

  However, to allow the use of the kernel trick, we'll use the dual
  Lagrangian form of the optimization again.
  
\end{frame}


\begin{frame}{Support vector machines}{Non-separable data}

  Illustration of slack variables:

  \medskip

  \myfig{2.8in}{slack}{Bishop (2006), Figure 7.3}

\end{frame}


\begin{frame}{Support vector machines}{Non-separable data}

  The Lagrangian is
  \[ \begin{array}{rl}
    {\cal L}(\vec{w},b,\vec{\xi},\vec{\alpha},\vec{r}) =
  &\frac{1}{2}\vec{w}^\top \vec{w} + C\sum_{i=1}^m\xi_i -\\
  &\sum_{i=1}^m\alpha_i \left[ y^{(i)}(\vec{x}^{(i)\top}\vec{w}+b)-1+\xi_i\right] -
  \sum_{i=1}^mr_i\xi_i . \end{array} \]

  \medskip

  Taking derivatives with respect to $\vec{w}$ and $b$, setting them
  to 0, substituting them in, and simplifying, we can obtain the dual
  optimization problem
  \[ \begin{array}{rl} \max_{\vec{\alpha}} & W(\vec{\alpha}) = \sum_{i=1}^m\alpha_i
    - \frac{1}{2}\sum_{i,j=1}^my^{(i)}y^{(j)}\alpha_i\alpha_j
    \left< \vec{x}^{(i)},\vec{x}^{(j)}\right> \\
    \text{such that} & 0 \le \alpha_i \le C, i \in 1..m \\
    & \sum_{i=1}^m \alpha_i y^{(i)} = 0 .\end{array} \]

\end{frame}


\begin{frame}{Support vector machines}{Non-separable data}

  Amazingly, the $\xi_i$ and $r_i$ drop out of the optimization.

  \medskip

  This leaves the original objective function for the $\alpha_i$ with
  an a new constraint that $\alpha_i \le C$.

  \medskip

  So if we can learn the $\alpha_i$ for the separable case, all we need
  to do for outliers and overlapping data is put a limit on $\alpha_i$.

  \medskip

  Larger $C$ makes the final model less tolerant toward training
  points crossing the decision boundary.

  \medskip

  Smaller $C$ will allow more training points to cross the boundary
  in order to get a good overall fit to the data.
  
\end{frame}


\begin{frame}{Support vector machines}{Non-separable data}

  The KKT dual-complementarity conditions give the following:
  \begin{eqnarray}
    \alpha_i = 0 & \implies & y^{(i)}(\vec{w}^\top\vec{x}^{(i)}+b) \ge 1 \nonumber \\
    \alpha_i = C & \implies & y^{(i)}(\vec{w}^\top \vec{x}^{(i)}+b) \le 1 \nonumber \\
    0 < \alpha_i < C & \implies & y^{(i)}(\vec{w}^\top \vec{x}^{(i)}+b) = 1 \nonumber
  \end{eqnarray}

  So the initial assumption of linearly separable data turns out not to be a big deal!

  \medskip

  One thing to note, however: the parameter $C$ is a \alert{free
    parameter} or \alert{hyperparameter} that must be set to some
  reasonable value, usually through a cross validation experiment
  (more on this later).

  \medskip

  So now all we need is an efficient algorithm to solve the dual
  optimization problem.
  
\end{frame}


\begin{frame}{Support vector machines}{Optimizing the dual}

  We begin with the \alert{coordinate ascent} algorithm.

  \medskip

  Suppose the goal is to solve the (unconstrained for now) optimization
  problem
  \[ \max_{\vec{\alpha}} W(\alpha_1,\alpha_2,\ldots,\alpha_m) .\]
  Rather than gradient ascent or Newton's method, we might now try
  to optimize the parameters one by one:

  \begin{tabbing}
    xxx \= xxx \= \kill
    While not converged do \\
    \> for $i = 1, \ldots, m$ do \\
    \> \> $\alpha_i \leftarrow \argmax_{\hat{\alpha}_i} W(\alpha_1,
    \ldots,\alpha_{i-1},\hat{\alpha}_i,\alpha_{i+1},\ldots,\alpha_m)$
  \end{tabbing}

\end{frame}


\begin{frame}{Support vector machines}{Optimizing the dual}

  Visualization of coordinate ascent:
  \myfig{3in}{ng-coord-ascent}{Ng, CS 229 lecture notes, set 3}
  
\end{frame}


\begin{frame}{Support vector machines}{Optimizing the dual}

  To adapt coordinate ascent to the SVM dual optimization problem,
  consider that we have a set of $\alpha_i$'s that statisfy the constraints.

  \medskip

  Can we hold $\alpha_2,\ldots$ fixed while optimizing with respect to
  $\alpha_1$?

  \medskip

  No, because the constraint $\sum_{i=1}\alpha_iy^{(i)} = 0$ means that if we
  increase (decrease) an $\alpha_i$ with $y^{(i)}=-1$ we have to make a
  correspnding decrease (increase) in an $\alpha_i$ with $y^{(i)}=-1$
  or a corresponding increase (decrease) in an $\alpha_i$ with $y^{(i)}=+1$.
  
\end{frame}


\begin{frame}{Support vector machines}{Optimizing the dual}

  The SMO algorithm by John Platt (1998) therefore updates \alert{pairs}
  of $\alpha_i$'s at a time:

  \begin{tabbing}
    xxx \= xxx \= xxx \kill
    While not converged \\
    \> Select a pair $(\alpha_i,\alpha_j)$ \\
    \> Optimize $W(\vec{\alpha})$ with respect to $\alpha_i$ and $\alpha_j$.
  \end{tabbing}

  \medskip

  The convergence criterion is whether the KKT conditions are
  satisfied within some tolerance.
  
\end{frame}


\begin{frame}{Support vector machines}{Optimizing the dual}

  Suppose we've selected $\alpha_1$ and $\alpha_2$ as the parameters
  to optimize on one iteration.

  \medskip

  Clearly, due to the constraint that $\sum_i \alpha_iy^{(i)} = 0$, we can
  write
  \[ \alpha_1 y^{(1)} + \alpha_2 y^{(2)} = - \sum_{i=3}^m \alpha_i y^{(i)}, \]
  or, equivalently for some constant $\zeta$:
  \[ \alpha_1 y^{(1)} + \alpha_2 y^{(2)} = \zeta. \]
  
\end{frame}


\begin{frame}{Support vector machines}{Optimizing the dual}

  The $\alpha_i$'s are already constrained to lie within $0 \le \alpha_i \le C$.

  \medskip

  Imagine this constraint as a square. The new constraint means $\alpha_1$ and
  $\alpha_2$ must lie on a line:

  \medskip

  \myfig{2.5in}{ng-smo-constraints}{Ng, CS 229 lecture notes set 3}
  
\end{frame}


\begin{frame}{Support vector machines}{Optimizing the dual}

  We can now rewrie $\alpha_1$ in terms of $\alpha_2$:
  \[ \alpha_1 = (\zeta-\alpha_2y^{(2)})y^{(1)}. \]
  Substituting this into our objective function, and keeping
  $\alpha_3,...\alpha_m$ fixed, we obtain a quadratic function in
  $\alpha_2$ that can be solved by setting the partial derivative to
  0.

  \medskip

  To ensure that the new $\alpha_2$ satisfies the box constraint, we
  just ``clip'' it so that it and $\alpha_1$ will lie in the range $0
  \le \alpha_2 \le C$.

  \medskip

  You now know enough to read the SMO paper by John Platt for more detail
  and a nice pseudocode description of the full algorithm.
  
\end{frame}


\begin{frame}{Support vector machines}{Production implementations}

  LIBSVM is probably the best open-source implementation of SVM
  model training and prediction:
  \begin{itemize}
  \item It implements an algorithm similar to SMO.
  \item It has wrappers for every major programming language and a
    command-line interface on Linux (\texttt{apt-get install
      libsvm-tools}).
  \item It is included in scikit-py and OpenCV.
  \item \url{https://www.csie.ntu.edu.tw/~cjlin/libsvm/}
  \end{itemize}

  \medskip
  
  Another popular open source library is SVM$^{light}$ from Cornell.
  \begin{itemize}
    \item \url{https://www.cs.cornell.edu/people/tj/svm_light/}
  \end{itemize}

  \medskip

  \alert{Hyperparameter selection} ($C$, other parameters depending on
  the kernel) is extremely important with SVMs!  See LIBSVM's
  \texttt{easy.py} script to automate the process.

\end{frame}

\def\show{0}
\ifnum\show=1
\fi

\end{document}

